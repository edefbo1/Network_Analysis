\PassOptionsToPackage{unicode}{hyperref}
\PassOptionsToPackage{hyphens}{url}
%
\documentclass[12pt, a4paper]{article}
\usepackage[a4paper,margin=1in]{geometry}
\setlength\parindent{0pt}
\usepackage{mathptmx}
\usepackage{amsmath,amssymb}
\usepackage[T1]{fontenc}
\usepackage[utf8]{inputenc}
\usepackage{textcomp}
\usepackage{graphicx}
\usepackage{hyperref}
\usepackage{float}
\usepackage{wrapfig}
\usepackage{enumitem}
\usepackage{subfig}
\usepackage{titlesec}
\usepackage{placeins}
\setcounter{secnumdepth}{4} 
\titleformat{\paragraph}
  {\normalfont\normalsize\bfseries}{\theparagraph}{1em}{}
\renewcommand{\theparagraph}{\thesubsubsection.\arabic{paragraph}}

\begin{document}
\title{Network Analysis of Musical Collaborations on Spotify: The Italian Music Scene}
\author{Federico Borci, Computer science, 0001130010
  \\Federica Santisi, Computer science, 0001186853
  \\Giorgia Pirelli, Computer science, 0001176116}
\date{January~9\textsuperscript{th},~2026}

\maketitle
\section{Introduction}
\label{introduction}

The music industry has undergone radical transformations in the last two
decades, shifting from a traditional model based on physical record sales to a
digital ecosystem dominated by streaming platforms. Spotify, launched in 2008,
has become the leading global music streaming service, with over 500 million
active users and a catalog exceeding 100 million tracks.

In this new landscape, musical collaborations, commonly known as "featurings,"
have assumed a central role. Whereas in the past collaborations were relatively
rare events often limited to special projects, today they represent a
fundamental strategy for artists at every level of popularity. Collaborations
enable artists to reach new audiences, experiment with different musical
genres, increase their visibility on streaming platforms, and forge strategic
connections within the music industry.

Social Network Analysis offers powerful methodological tools for studying
these relational dynamics. By representing artists as nodes and their
collaborations as edges, it is possible to construct a network that captures
the complexity of interactions in the contemporary musical landscape.

This study focuses on the analysis of collaborations among artists on Spotify,
with a particular emphasis on the Italian music scene as the primary case
study.

\subsection*{Research Questions}
From a network perspective, this study addresses the following research
questions:
\begin{enumerate}[leftmargin=*, itemsep=4pt]

  \item \textbf{What are the main structural and connectivity properties of the Italian
  musical collaboration network?}
        Global network metrics and degree distribution.

  \item \textbf{How does the Italian musical scene compare with other major countries?}
        Cross-country comparison of network structure and scale.

  \item \textbf{Which artists occupy central positions, and how do collaborative strategies
  differ between emerging and established artists?}
        Centrality analysis and artist stratification.

  \item \textbf{Does the network exhibit assortative mixing and community structures related
  to musical genres and artist attributes?}
        Assortativity and community detection analysis.

\end{enumerate}

\section{Datasets}
\label{datasets}

The initial dataset used for this analysis was downloaded from
Kaggle \cite{kaggle-dataset}. After careful
inspection, it was considered a reliable source, as it is derived from publicly
available data provided by the Spotify API and already pre-processed to
represent collaboration relationships among artists. The resulting dataset is
modeled as an undirected, attribute-enriched network, where node-level metadata
enable the investigation of social, cultural, and structural mechanisms.

The dataset consists of two main files:
\begin{itemize}
  \item \textbf{nodes.csv} contains the nodes of the graph, where each node
  represents an artist. For each artist, the file provides a unique identifier,
  name, number of Spotify followers, a popularity score,
  a list of associated musical genres, and information about chart hits in
  different countries.
  \item \textbf{edges.csv} contains the edges of the graph, representing
  collaborations between pairs of artists. Each row specifies a source artist
  and a target artist through their corresponding identifiers.
\end{itemize}

The initial objective was to enrich the graph with additional artist-level
information, specifically \textbf{nationality} and \textbf{dominant musical
genre}, in order to enable more in-depth social and cultural analyses of the
network.


\subsection{Artist Nationality Enrichment}
\label{subsec:nationality}
Nationality is not directly observable in the original dataset and is therefore
treated as an inferred node attribute. The enrichment strategy was designed to
maximize coverage while minimizing systematic misclassification bias.
To associate a nationality with each artist, two complementary strategies were
adopted:

\begin{enumerate}
  \item \textbf{Inference based on musical genre.}
        In the first approach, nationality was inferred by analyzing the associated musical genres. For instance, an artist labeled with the genre \texttt{italian hip hop} was classified as \emph{Italian}. This method allowed the automatic assignment of nationality to a substantial subset of artists; however, it was not applicable in all cases, as many genres do not contain explicit geographical references.

  \item \textbf{Completion using an external dataset (MusicBrainz).}
        For artists whose nationality could not be inferred in the first step, data from the \texttt{MusicBrainz Dump (mbdump)} \cite{musicbrainz} were integrated. A direct matching based solely on artist names posed significant challenges due to the presence of homonyms with different nationalities. To mitigate this issue, the integration was performed exclusively on artists that remained unclassified after the first inference step.
\end{enumerate}

\subsection{Addition of Musical Genre}

To associate one or more musical genres with artists in the dataset, a
multi-level procedure was designed with the goal of maximizing coverage while
preserving semantic consistency. Each stage operates only on artists that
remained unclassified after the previous step.

\begin{itemize}
  \item \textbf{Direct mapping of Spotify genres.}
        Spotify-specific genres were first normalized and mapped to a limited set of musical macro-categories (e.g., \emph{Pop}, \emph{Rock}, \emph{Hip Hop / Rap}, \emph{Electronic / Dance}) using lexical and keyword-based rules.

  \item \textbf{AI-assisted completion.}
        Genres that could not be mapped automatically were classified through an AI-assisted process, which assigned them to the predefined macro-categories based on semantic similarity.

  \item \textbf{Inference via artistic collaborations.}
        For artists still lacking a genre, the collaboration network was exploited by assigning the most frequent genres among direct collaborators. When necessary, this inference was extended using a Breadth-First Search (BFS) up to three levels.

  \item \textbf{Popularity-based inference.}
        Remaining unclassified artists were analyzed using quantitative indicators such as follower count and popularity, inferring the most likely genres based on patterns observed in the dataset.

  \item \textbf{Global fallback assignment.}
        In the few remaining cases, a fallback strategy based on the most common genres in the dataset was applied.
\end{itemize}

This multi-level strategy enabled the creation of a dataset that is complete
and consistent from a musical genre perspective, minimizing arbitrary
assignments and integrally exploiting semantic, structural, and quantitative
information. The final result constitutes a solid foundation for the subsequent
network and artistic community analyses.

\section{Validity and Reliability}
\label{validity-reliability}
The dataset represents a realistic abstraction of musical collaborations on
Spotify, derived from Spotify API data and modeled as reciprocal relationships
between artists. This modeling choice aligns with network theory and is suitable
for analyzing structural properties of the collaboration ecosystem, while
abstracting from semantic and contractual aspects.

Limitations mainly arise from platform-specific biases, partial data
incompleteness, and approximations in inferred attributes such as nationality
and dominant genre. These effects are mitigated through conservative,
rule-based, multi-stage enrichment procedures, supporting construct validity.

Reliability is ensured by a rigorous and transparent analytical methodology
implemented using established and reliable software tools. All preprocessing
and analysis steps are deterministic and reproducible. The only partially
non-deterministic component is AI-assisted genre classification; however, its
outputs are fixed prior to analysis, preserving reproducibility and limiting its
impact on validity.


\section{Measures and Results}
\label{measures}
\subsection{General Analysis of the Italian Musical Collaboration Network}

In order to outline the key structural differences and to obtain an overview of
the topology and internal dynamics of the Italian musical collaboration
network, a series of general metrics were calculated, as reported in Table
\ref{tab:metrics_italia}. 

\begin{table}[h]
  \centering
  \caption{Structural characteristics of the Italian musical collaboration network}
  \label{tab:metrics_italia}
  \begin{tabular}{|l|c|}
    \hline
    \textbf{Parameter}             & \textbf{Value} \\
    \hline
    Total nodes (artists)          & 1,656          \\
    Total edges (collaborations)   & 4,307          \\
    Connected components           & 16             \\
    Maximum node degree            & 114            \\
    Average node degree            & 5.20           \\
    Density                        & 0.00314        \\
    Diameter                       & 10             \\
    Average shortest path length   & 4.14           \\
    Average clustering coefficient & 0.119          \\
    Transitivity                   & 0.128          \\
    \hline
  \end{tabular}
\end{table}
\bigskip
The sixteen connected components suggest the existence of isolated groups, corresponding to niches or artistic communities with limited contact with the rest of the national ecosystem. On average, each artist is connected to about five colleagues in the network. 

However, the degree distribution is highly heterogeneous: the presence of a node with degree 114 reveals a central \textit{hub} of major importance, while 54.6\% of artists have a degree of 1, thus collaborating with only one other artist. The extremely low density (approximately 0.31\% of possible connections are realised) confirms the \textbf{sparse} nature of the network. The network diameter is 10, indicating that any two artists can be connected through at most 10 intermediate collaborations. 

Despite the low density, the network presents a modest average path length (4.14), indicating that artists are connected through few intermediate steps.

The values of the average clustering coefficient (0.119) and transitivity
(0.128) are moderate and close to each other. This indicates a measurable,
though not dominant, tendency towards \textbf{triadic closure}: two
collaborators of the same artist have approximately a 12\% probability of
having collaborated with each other in turn. This local cohesion fosters the
formation of cohesive artistic circles and partially clustered communities,
contributing to the stability of collaborative relationships and the sharing of
artistic practices within subgroups, while maintaining sufficient openness to
allow connections between different communities.

\subsubsection{Comparative Analysis with Major European Countries}

To situate the Italian results within a broader continental context, the
analysis was extended to major European countries. This approach enables a
comparative evaluation of collaborative dynamics. Table
\ref{tab:comparative_results_eu} presents the comparison between Italy and the
top 5 European countries by network size.

\begin{table}[h]
  \centering
  \caption{Comparison with the top 5 European countries by network size}
  \label{tab:comparative_results_eu}
  \resizebox{\textwidth}{!}{%
    \begin{tabular}{|l|c|c|c|c|c|c|}
      \hline
      \textbf{Metric}      & \textbf{Italy (3th)} & \textbf{United Kingdom (1st)} & \textbf{Germany (2nd)} & \textbf{France (4th)} & \textbf{Netherlands (5th)} \\
      \hline
      Total nodes          & 1,656          & 3,290                         & 2,706                  & 1,643                 & 1,420                      \\
      Total edges          & 4,307          & 7,532                         & 5,929                  & 4,754                 & 5,143                      \\
      Average degree       & 5.20           & 4.58                          & 4.38                   & 5.79                  & 7.24                       \\
      Connected components & 16             & 70                            & 32                     & 27                    & 12                         \\
      Average clustering   & 0.119          & 0.062                         & 0.120                  & 0.113                 & 0.151                      \\
      Density              & 0.00314        & 0.001392                      & 0.001618               & 0.003525              & 0.005107                   \\
      \hline
    \end{tabular}%
  }
\end{table}

The analysis reveals significant structural differences among countries. The
\textbf{Netherlands} exhibit the highest \textbf{average degree} in Europe
(7.24) despite ranking only fifth in terms of number of artists. Their
\textbf{average clustering coefficient} (0.151) is the second highest, and the
percentage of hub artists (17.8\%) is the maximum in Europe. The \textbf{United Kingdom}, despite its larger size
(3,290 artists), presents the lowest \textbf{average clustering coefficient}
(0.062) and the highest fragmentation (70 \textbf{connected components}),
reflecting a vast but segmented market. It also shows a significantly lower
network density (0.001392) compared to that observed for Italy (0.003143).
Italy's position within the European context is intermediate. With 1,656
artists, the Italian network ranks third in size, but its \textbf{average
  degree} (5.20) is lower only than that of the Netherlands and France,
indicating an intense propensity for collaboration per artist. Its
\textbf{density} (0.00314) is significantly higher than that of much larger
networks such as the United Kingdom and Germany, suggesting a more compact
ecosystem. The moderate fragmentation (16 components) and the \textbf{average
  clustering} (0.119) place Italy in a balanced position between the high
cohesion of Poland and Greece and the low cohesion of the United Kingdom.

\begin{figure}[htbp]
  \centering
  \includegraphics[width=0.95\textwidth]{network_size_eu.png}
  \caption{Comparative multidimensional analysis of musical collaboration networks at the European level.}
  \label{fig:network_analysis_eu}
\end{figure}

Figure \ref{fig:network_analysis_eu} presents a comparative multidimensional
analysis of musical collaboration networks at the European level. The analysis
synthesises three critical aspects of the structure of European networks: size,
collaborative intensity, and the relationship between these variables. The
histograms present the top 15 European countries for two distinct metrics: on
the left, the number of artists; in the centre, the total number of
collaborations. In the latter, it is observed that the \textbf{United Kingdom}
(7,532 collaborations) and \textbf{Germany} (5,929 collaborations) rank at the
top for the overall number of collaborations; this result is primarily
attributable to their high number of nodes. The \textbf{Netherlands} (5,143
collaborations) represent the most relevant case: despite being fifth in number
of artists, they rank third in collaboration volume, surpassing both Italy
(4,307) and France (4,754). This indicates intense collaborative activity.
\textbf{Poland} (3,391 collaborations) shows surprising intensity, surpassing
countries with larger networks such as Sweden (2,644) and Finland (2,329).

The chart on the right explores the fundamental relationship between a
network's size (number of artists) and its activity (number of collaborations).
The relationship between the number of artists and the total number of
collaborations was analysed using linear regression estimated by the least
squares method, considering all European countries included in the study.
Countries
positioned \textbf{above the regression line} (Netherlands, Poland)
represent hyper-collaborative ecosystems, where the volume of interactions
systematically exceeds expectations given their dimensions. Countries
\textbf{close to the line} (Italy, France, Germany, United Kingdom) follow an
approximately linear relationship between collaborative activity and size. 
\subsubsection{Comparative Analysis with Major Non-European Countries}

To situate the Italian musical ecosystem in a broader global context, a
comparative analysis was conducted, extended to 49 non-European countries.
Table \ref{tab:comparative_results_global} presents the comparison between
Italy and the top 5 non-European countries by network size.

\begin{table}[h]
  \centering
  \caption{Comparison with the top 5 non-European countries by network size}
  \label{tab:comparative_results_global}
  \resizebox{\textwidth}{!}{%
    \begin{tabular}{|l|c|c|c|c|c|c|}
      \hline
      \textbf{Metric}      & \textbf{Italy} & \textbf{United States (1st)} & \textbf{Brazil (2nd)} & \textbf{India (3rd)} & \textbf{Japan (4th)} & \textbf{Mexico (5th)} \\
      \hline
      Total nodes          & 1,656          & 6,217                        & 1,859                 & 999                  & 890                  & 773                   \\
      Total edges          & 4,307          & 14,860                       & 5,547                 & 2,834                & 1,103                & 1,665                 \\
      Average degree       & 5.20           & 4.78                         & 5.97                  & 5.67                 & 2.48                 & 4.31                  \\
      Connected components & 16             & 111                          & 3                     & 2                    & 38                   & 4                     \\
      Average clustering   & 0.119          & 0.082                        & 0.173                 & 0.181                & 0.042                & 0.156                 \\
      Density              & 0.00314        & 0.000769                     & 0.003212              & 0.005685             & 0.002788             & 0.005580              \\
      \hline
    \end{tabular}%
  }
\end{table}

\bigskip

The \textbf{United States} exhibit the largest network, with 6,217
artists and 14,860 collaborations, over 3.7 times larger than the Italian
network. This scale is accompanied by very low density (0.000769) and a high
number of connected components (111), indicating a highly diversified and
fragmented music market. Despite its smaller size, Italy shows a higher average
degree (5.20 vs.\ 4.78), suggesting more intense collaborative activity.
\textbf{Brazil} and \textbf{India} emerge as large and highly cohesive musical
systems, characterised by few connected components, and strong
clustering.

In terms of collaborative intensity, Italy ranks among the most active
countries globally, surpassing the United States, Japan, Mexico
and South Korea in average degree. Several countries display hyper-collaborative
structures: \textbf{Puerto Rico} represents the most extreme case, while
\textbf{Brazil}, \textbf{India}, and \textbf{South Korea} combine relatively
large networks with strong internal interconnectedness.

The analysis of network density reveals clear regional patterns. The analysed
\textbf{African} countries (including Ghana and Algeria) exhibit very small but
extremely dense networks, reflecting highly compact music scenes. \textbf{Latin
American} countries show intermediate densities, still substantially higher
than those observed in European and North American networks. Italy (0.00314)
occupies an intermediate position: denser than the United States, comparable to
Brazil, but less dense than India and Mexico, indicating a balanced trade-off
between network size and cohesion.

Finally, the highest \textbf{clustering coefficients} are observed in
\textbf{Puerto Rico}, \textbf{Ghana}, and the \textbf{Dominican Republic},
followed by \textbf{India} and \textbf{Brazil}, which combine large network size
with strong local cohesion. Italy displays a moderate clustering value (0.119),
higher than those of the United States and Japan but lower than the major
emerging countries. This intermediate position suggests a musical ecosystem
that effectively balances local cohesion and structural openness.


\subsection{Nodes}
\subsubsection{Centrality Measures}

To identify the structurally most important artists in the collaboration
network, four centrality measures were calculated. The objective is to
understand which artists occupy strategic positions and how these positions
manifest themselves through different aspects of the network's structure.

\paragraph{Degree Centrality}

\textbf{Degree centrality} quantifies the number of direct connections of a node, normalized by the maximum possible number of connections. In the analyzed musical network, this measure represents the number of distinct artists with whom a given artist has collaborated.

The implementation calculates both the absolute degree and the normalized
degree centrality. The results reveal a strongly asymmetric distribution: the
average degree centrality is 0.0031, while the maximum value reaches 0.0689.
This distribution indicates that the majority of artists maintain a limited
number of collaborations, while a small group of nodes concentrates a
significantly high number of connections.

\bigskip
\begin{table}[h]
  \centering
  \caption{Top 3 artisti per Degree Centrality}
  \label{tab:degree-centrality}
  \begin{tabular}{c l r}
    \hline
    \textbf{Posizione} & \textbf{Artista} & \textbf{Degree Centrality} \\
    \hline
    1                  & Guè              & 0.0689                     \\
    2                  & Andrea Bocelli   & 0.0622                     \\
    3                  & Clementino       & 0.0508                     \\
    \hline
  \end{tabular}
\end{table}

\FloatBarrier
Guè emerges as the most connected artist, with 114 distinct collaborations, confirming his central role in the Italian music network. Andrea Bocelli ranks second with 103 collaborations, highlighting his ability to connect across different musical genres. The strong presence of hip hop artists among the top positions reflects the collaborative nature of this genre and its structural relevance within the network. Ennio Morricone’s inclusion in the top ten introduces an element of stylistic and generational diversity.

\paragraph{Eigenvector Centrality}

\textbf{Eigenvector centrality} assigns importance not only to the quantity of connections but also to their quality: an artist has high eigenvector centrality if they are connected to other artists who themselves occupy central positions in the network. The algorithm converges iteratively, assigning each node a score proportional to the sum of the scores of its adjacent nodes.

The results show a highly concentrated distribution, with an average value of
0.0089 and a maximum of 0.2573, indicating the presence of a small and cohesive
core of highly influential artists.

\begin{table}[h]
  \centering
  \caption{Top 3 artists by Eigenvector Centrality}
  \label{tab:eigenvector-centrality}
  \begin{tabular}{c l r}
    \hline
    \textbf{Rank} & \textbf{Artist} & \textbf{Eigenvector Centrality} \\
    \hline
    1             & Guè             & 0.2573                          \\
    2             & Gemitaiz        & 0.2069                          \\
    3             & Emis Killa      & 0.1904                          \\
    \hline
  \end{tabular}
\end{table}

\FloatBarrier
Guè clearly dominates the ranking, with a value
approximately  0.2573 higher than the second-ranked artist, confirming his central role within the network. The ranking is exclusively composed of hip hop artists, revealing a genre-specific core that defines the structural center of the collaboration network. Artists such as Gemitaiz, Marracash, Jake La Furia, Don Joe, MadMan, and Lazza form a densely interconnected group. Notably, Andrea Bocelli is absent from this ranking despite his high degree centrality, suggesting that his collaborations are less embedded in the network’s central core.

\begin{figure}[H]
  \centering
  \includegraphics[width=0.25\linewidth]{images/eigenvector.pdf}
  \caption{Gephi representation of Eigenvector}
  \label{fig:placeholder}
\end{figure}

\paragraph{Closeness Centrality}
\textbf{Closeness centrality} measures how close a node is to all other nodes in the network by computing the inverse of the average shortest-path distance. In the context of musical collaborations, an artist with high closeness centrality can reach any other artist through a small number of intermediaries, indicating a structurally advantageous position for communication and interaction.

The distribution shows an average value of 0.2336 and a maximum of 0.3677,
suggesting that even the most central artists require only a few steps to
access the entire network.

\begin{table}[h]
  \centering
  \caption{Top 3 artists by Closeness Centrality}
  \label{tab:closeness-centrality}
  \begin{tabular}{c l r}
    \hline
    \textbf{Rank} & \textbf{Artist} & \textbf{Closeness Centrality} \\
    \hline
    1             & Guè             & 0.3677                        \\
    2             & Clementino      & 0.3576                        \\
    3             & Gemitaiz        & 0.3537                        \\
    \hline
  \end{tabular}
\end{table}

\FloatBarrier

Guè again occupies the top position, confirming his
optimal placement within the network. In addition to the artists reported in
the table, relatively high closeness centrality values are also observed for
artists such as Clementino and J-AX, who appear more prominent in this metric
than in eigenvector centrality. This pattern suggests a bridging role between
different regions of the network rather than strong membership in its densest
core. The presence of artists like Elisa and Rocco Hunt further highlights the
ability of the network to support efficient connectivity across genres.
Overall, closeness centrality identifies artists who facilitate rapid
information flow and collaboration opportunities across the entire musical
ecosystem.

\paragraph{Betweenness Centrality}
Betweenness centrality measures how often a node lies on the shortest paths
between other pairs of nodes. In a collaboration network, artists with high
betweenness act as intermediaries connecting different clusters, even if they
do not have many direct collaborations. For computational efficiency, the
metric was approximated by sampling 1,000 nodes.

The distribution is highly skewed, with an average value of 0.0018 and a
maximum of 0.1250, indicating that only a small number of artists exert
substantial control over the flow of connections between different network
communities. Andrea Bocelli clearly dominates this metric, highlighting his
role as a structural bridge between classical, pop, and other musical genres.
In addition to the artists reported in Table~\ref{tab:betweenness-centrality},
relatively high betweenness values are also observed for artists such as Ennio
Morricone, DJ Matrix, Jovanotti, and Cristina D’Avena, confirming that
betweenness centrality captures a bridging role that is distinct from local or
popularity-based measures. Clementino further stands out by combining high
centrality across multiple metrics, effectively linking the hip hop core with
other regions of the network.

\begin{table}[h]
  \centering
  \caption{Top 3 artists by Betweenness Centrality}
  \label{tab:betweenness-centrality}
  \begin{tabular}{c l r}
    \hline
    \textbf{Rank} & \textbf{Artist} & \textbf{Betweenness Centrality} \\
    \hline
    1             & Andrea Bocelli  & 0.1250                          \\
    2             & Clementino      & 0.0910                          \\
    3             & Guè             & 0.0642                          \\
    \hline
  \end{tabular}
\end{table}

\subsubsection{Assortativity}

Assortativity analysis examines connection tendencies based on specific node
attributes. Three forms of assortativity were calculated to characterize
collaboration patterns in the Italian musical network.

\paragraph{Degree Assortativity}

The degree assortativity coefficient of the network is \textbf{-0.1052},
indicating a slightly \textbf{disassortative} structure. This negative value
shows that highly connected artists tend to collaborate with less connected
artists, rather than forming tight clusters among themselves.

In the musical network, this pattern suggests that hubs do not form an isolated
core but include less prolific artists in their collaborations. This may
reflect several dynamics: established artists providing visibility to emerging
talent, producers and featuring artists working across a wide spectrum of
collaborators, or a generally open music scene where past collaborations do not
limit future opportunities. Although moderate, the negative assortativity
supports a structure that promotes mobility and access for less central
artists, contrasting with rigid hierarchical networks where only the most
successful figures collaborate among themselves.

\paragraph{Followers Assortativity}

The followers assortativity coefficient is \textbf{0.0724}, a positive but
near-zero value, indicating a very weak tendency towards assortative patterns
based on popularity. This metric measures whether artists with a similar number
of Spotify followers preferentially collaborate with each other.

The result suggests that popularity has limited influence on collaboration
patterns. Highly followed artists do not exclusively collaborate with similarly
popular artists, nor do less popular artists remain isolated. Categorizing
edges into High-High, Low-Low, and High-Low collaborations using the median
follower count shows a relatively balanced distribution, with both intra- and
inter-popularity collaborations occurring.

This indicates a fluid Italian music scene, where fame does not strongly
constrain partnerships. While the slightly positive coefficient hints at a weak
homophilic tendency very popular artists collaborating somewhat more frequently
with other popular artists it is not strong enough to create significant
structural barriers within the network.

\paragraph{Genre Assortativity and Modularity}

The musical genre assortativity coefficient is \textbf{0.4778}, a clearly
positive value indicating a strong tendency for artists to collaborate within
the same or related genres. The modularity is \textbf{Q = 0.2987}, moderate
according to the interpretative scale, showing that while genre strongly
influences collaborations, cross-genre connections still maintain global
network cohesion.

\begin{table}[h]
  \centering
  \caption{Top 10 genre pairs in collaborations}
  \label{tab:genre-pairs}
  \begin{tabular}{c l r l}
    \hline
    \textbf{Rank} & \textbf{Genre 1}        & \textbf{N. collab.} & \textbf{Genre 2} \\
    \hline
    1             & Hip Hop / Rap           & 1829                & (intra-genere)   \\
    2             & Pop                     & 819                 & (intra-genere)   \\
    3             & Hip Hop / Rap           & 596                 & Pop              \\
    4             & Hip Hop / Rap           & 104                 & Indie            \\
    5             & Elettronica / Dance     & 85                  & (intra-genere)   \\
    6             & Classica / Orchestrale  & 72                  & (intra-genere)   \\
    7             & Elettronica / Dance     & 62                  & Pop              \\
    8             & Indie                   & 45                  & Pop              \\
    9             & Soundtrack / Film Score & 37                  & (intra-genere)   \\
    10            & Elettronica / Dance     & 34                  & Hip Hop / Rap    \\
    \hline
  \end{tabular}
\end{table}

Hip Hop /
Rap dominates with 1829 intra-genre collaborations, followed by Pop with 819.
The Hip Hop / Rap + Pop combination is the most frequent cross-genre connection
(596 edges), acting as a structural bridge between the two major genres. Other
notable cross-genre collaborations include Hip Hop / Rap + Indie and
Elettronica / Dance + Pop, though less frequent. Smaller yet cohesive
communities are found in Classica / Orchestrale, Elettronica / Dance, and
Soundtrack / Film Score.

Overall, the results reveal a pattern of genre specialization with selective
cross-genre interactions. While artists tend to collaborate predominantly
within their stylistic boundaries, established bridges, especially between hip
hop and pop, ensure network connectivity. Hip Hop / Rap’s high internal density
confirms its central role and highly interconnected internal ecosystem.

\subsection{Network}
\subsubsection{Community Detection}
Community detection was applied to assess genre-based collaboration patterns.
Two methods were considered, namely the Louvain algorithm and the Edge
Betweenness approach. For each method, the number of detected communities and
their genre purity were evaluated.

\noindent
\subsubsection*{Louvain}

\begin{wrapfigure}{r}{0.35\textwidth} % r = right, larghezza 50%
  \centering
  \vspace{-15pt} % Regola per allineare l'immagine alla prima riga di testo
  \includegraphics[width=\linewidth]{images/louvain.pdf}
  \caption{Community structure obtained by applying the Louvain algorithm in Gephi.}
  \label{fig:louvain}
  \vspace{-25pt} % Riduce lo spazio sotto la caption
\end{wrapfigure}

The Louvain algorithm identified a total of 34 communities, revealing a
relatively fragmented network structure. Several communities exhibit high genre
homogeneity, particularly for \emph{Hip Hop / Rap} and \emph{Pop}, with purity
values exceeding 0.6 and reaching 1.0 in smaller clusters. This pattern is
visually reflected in the Louvain-based Gephi layout
(Figure~\ref{fig:louvain}). Overall, this indicates that collaborations are not
strictly constrained by genre boundaries, especially within larger communities.

\subsubsection*{Louvain versus Genre Assortativity}
The community detection results can be interpreted using genre assortativity
and modularity measures. The genre assortativity coefficient
($r = 0.4778$) indicates a clear homophilic tendency, with artists more likely
to collaborate within the same or closely related macro-genres.

The moderate modularity value ($Q = 0.2987$) suggests that genre does not
induce a strong structural separation, due to frequent cross-genre
collaborations, notably between \emph{Hip Hop / Rap} and \emph{Pop}.

Overall, the network displays a pattern of \emph{selective mixing}: local
genre-based homophily coexists with cross-genre ties, resulting in a structure
that is both cohesive and interconnected.

\subsubsection*{Edge Betweenness}

The Edge Betweenness (Girvan--Newman) algorithm produced 17 communities,
resulting in a coarser partition than Louvain. Communities are generally less
pure, with genre purity often below 0.5, highlighting the role of bridge nodes
and cross-genre collaborations rather than clear genre separation.
\newpage
\subsubsection{Degree Distribution}
\noindent
\begin{wrapfigure}{r}{0.35\textwidth} % r = destra, larghezza 45%
  \centering
  \vspace{-20pt} % Riduce lo spazio superiore per allineare l'immagine al testo
  \includegraphics[width=\linewidth]{images/degree.pdf}
  \captionof{figure}{Artist collaboration network visualized in Gephi with node size proportional to degree.}
  \label{fig:degree_gephi}
  \vspace{-35pt} % Riduce lo spazio bianco sotto la didascalia
\end{wrapfigure}
The analysis of the degree distribution provides insight into the global
structure of the artist collaboration network. The minimum degree of 1 reflects
the presence of artists involved in a single collaboration, while the maximum
degree of 114 highlights a small set of highly connected nodes acting as hubs.
The average degree of 5.20 indicates an overall sparse network.

This heterogeneous connectivity pattern is visible in the Gephi visualization
(Figure~\ref{fig:degree_gephi}), where node size is proportional to degree,
indicating a strongly right-skewed distribution. This structure is typical of
complex networks, where a small number of hubs play a central role in
maintaining global connectivity. \vspace{1em}

\begin{figure}[H]
  \centering
  \subfloat[Degree distribution in linear scale.\label{fig:dd_linear}]{
    \includegraphics[width=0.48\textwidth]{images/DD-AC.png}
  }
  \hfill
  \subfloat[Degree distribution in logarithmic scale.\label{fig:dd_log}]{
    \includegraphics[width=0.48\textwidth]{images/DDLG.png}
  }
  \caption{Degree distribution of the artist collaboration network.}
  \label{fig:degree_distribution}
\end{figure}

\section{Conclusion}
\label{conclusion}
This network analysis reveals that the Italian musical collaboration ecosystem 
on Spotify exhibits a complex and stratified structure. The network is characterized 
by \textbf{high sparsity} but efficient "small-world" connectivity. The degree distribution is highly 
heterogeneous: while the majority of artists  have only one collaboration, 
a small core of central hubs notably \textbf{Guè}, \textbf{Andrea Bocelli}, and 
\textbf{Clementino} concentrates a very high number of connections, supporting 
global cohesion.

Centrality measures show an influence concentration within a \textbf{cohesive
  core dominated by Hip Hop/Rap}, while
betweenness centrality highlights strategic bridge nodes connecting distinct stylistic communities such as classical, pop,
and hip hop. The Italian network occupies an \textbf{intermediate range
  globally} in size but stands
out for its \textbf{above-average collaborative propensity}, positioning itself above the European trend in the node-edge regression.

Assortativity analysis indicates \textbf{strong genre homophily}, with collaborations predominantly intra-genre, especially in hip hop
and pop. However, significant inter-genre bridges
particularly between Hip Hop/Rap and Pop ensure
\textbf{permeability and global cohesion}, as confirmed by moderate modularity. The structure is slightly disassortative by degree,
suggesting that highly connected artists frequently collaborate with less
central ones, fostering integration and mobility.

In summary, the Italian music scene on Spotify emerges as a \textbf{dynamic and
  balanced ecosystem}, characterized by a hyper-collaborative yet permeable core
that combines local specialization with cross-genre openness, establishing
itself as a relevant case study within the global digital music landscape.

\section{Critique}
\label{critique}

Despite the analysis having produced significant and interpretable results, it
is important to acknowledge the main intrinsic limitations of our
methodological approach and the data used.

\begin{itemize}[leftmargin=*]

  \item \textbf{Partial and platform-limited data}: The dataset includes exclusively collaborations officially registered on Spotify, excluding those occurring on other platforms (YouTube, SoundCloud), in live contexts, or in unofficial forms. This may lead to an underestimation of network density, especially for underground genres or emerging artists operating outside mainstream channels.

  \item \textbf{Automatic attribute inference}: Nationality and musical genre were assigned through automated procedures. While this maximized coverage, it may have introduced systematic errors or excessive simplifications.

  \item \textbf{Static versus dynamic analysis}: The network was treated as a fixed snapshot in time. A longitudinal approach would have allowed studying how communities form, how artists' centrality changes in response to events (album release, chart entry), and how collaborative strategies evolve across different career stages.

  \item \textbf{Simplicity of the network model}: The graph is unimodal (only artists) and undirected, and does not distinguish between occasional collaborations and stable partnerships. A richer representation, perhaps weighted according to the number of shared tracks or enriched with temporal metadata, would have enabled more accurate analysis.

  \item \textbf{Interpretive limits of the structural approach}: Quantitative analysis identifies connection patterns; one can measure that an artist connects two musical communities, but it is not known whether this is a conscious creative choice, a personal relationship, or a commercial strategy. Without integrating structural data with qualitative sources, such as interviews, lyric analysis, reconstruction of production contexts, or market dynamics, conclusions about "collaborative strategies" or the "social role" of artists remain plausible yet not overly detailed hypotheses.
\end{itemize}

\begin{thebibliography}{9}

  \bibitem{project-repo}
  \textit{Network Analysis Project Repository},
  GitHub,
  \url{https://github.com/edefbo1/Network_Analysis.git}

  \bibitem{gephi}
  Gephi Consortium,
  \textit{Gephi: an open source graph visualization and analysis software},
  Gephi.org,
  \url{https://gephi.org/}

  \bibitem{networkx}
  \textit{NetworkX Documentation and Tutorial},
  NetworkX.org,
  \url{https://networkx.org/documentation/stable/tutorial.html}

  \bibitem{kaggle-dataset}
  J. Freyberg,
  \textit{Spotify Artist Feature Collaboration Network},
  Kaggle Dataset,
  \url{https://www.kaggle.com/datasets/jfreyberg/spotify-artist-feature-collaboration-network/}

  \bibitem{musicbrainz}
  MusicBrainz Foundation,
  \textit{MusicBrainz Database — Download and Documentation},
  MusicBrainz.org,
  \url{https://musicbrainz.org/doc/MusicBrainz_Database/Download}

\end{thebibliography}

\end{document}