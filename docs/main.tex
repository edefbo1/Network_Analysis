\PassOptionsToPackage{unicode}{hyperref}
\PassOptionsToPackage{hyphens}{url}
%
\documentclass[12pt, a4paper]{article}
\usepackage[a4paper,margin=1in]{geometry}
\setlength\parindent{0pt}
\usepackage{mathptmx}
\usepackage{amsmath,amssymb}
\usepackage[T1]{fontenc}
\usepackage[utf8]{inputenc}
\usepackage{textcomp}
\usepackage{graphicx}
\usepackage{hyperref}
\usepackage{float}
\usepackage{enumitem}
\usepackage{subfig}
\usepackage{titlesec}
\usepackage{placeins}
\setcounter{secnumdepth}{4} 
\titleformat{\paragraph}
  {\normalfont\normalsize\bfseries}{\theparagraph}{1em}{}
\renewcommand{\theparagraph}{\thesubsubsection.\arabic{paragraph}}

\begin{document}
\title{Network Analysis of Musical Collaborations on Spotify: The Italian Music Scene}
\author{Federico , matricola
  \\Federica Santisi, matricola
  \\Giorgia Pirelli, matricola}
\date{December 2025}
\maketitle
\section{Introduction}
\label{introduction}

The music industry has undergone radical transformations in the last two
decades, shifting from a traditional model based on physical record sales to a
digital ecosystem dominated by streaming platforms. Spotify, launched in 2008,
has become the leading global music streaming service, with over 500 million
active users and a catalog exceeding 100 million tracks.

In this new landscape, musical collaborations, commonly known as "featurings,"
have assumed a central role. Whereas in the past collaborations were relatively
rare events often limited to special projects, today they represent a
fundamental strategy for artists at every level of popularity. Collaborations
enable artists to reach new audiences, experiment with different musical
genres, increase their visibility on streaming platforms, and forge strategic
connections within the music industry.

Social Network Analysis (SNA) offers powerful methodological tools for studying
these relational dynamics. By representing artists as nodes and their
collaborations as edges, it is possible to construct a network that captures
the complexity of interactions in the contemporary musical landscape.

This study focuses on the analysis of collaborations among artists on Spotify,
with a particular emphasis on the Italian music scene as the primary case
study.

\subsection*{Research Questions}
Through a network perspective, this work aims to answer fundamental questions
about the functioning of the Italian musical ecosystem and its positioning in
the international context:
\begin{enumerate}[leftmargin=*, itemsep=4pt]

  \item \textbf{What are the fundamental structural characteristics of the Italian musical
        collaboration network on Spotify?}  
        Global network metrics: number of nodes and edges, connected components,
        average and maximum degree, density, diameter, average shortest path length,
        clustering coefficient, and transitivity.

  \item \textbf{How does the Italian scene position itself relative to major European and
        non-European countries?}  
        Cross-country comparison using network size, number of collaborations, average
        degree, density, clustering coefficient, connected components, and node–edge
        linear regression.

  \item \textbf{Who are the most central and influential artists, and what structural role do
        they occupy?}  
        Degree, eigenvector, closeness, and betweenness centrality.

  \item \textbf{How are connections distributed within the network?}  
        Degree distribution analysis (linear and logarithmic scale).

  \item \textbf{What are the prevalent collaboration patterns? Do artists tend to collaborate
        with peers similar in terms of connectivity, popularity, or musical genre?}  
        Assortativity coefficients by degree, followers (popularity), and genre; intra- and
        inter-genre collaboration counts.

  \item \textbf{Do well-defined communities exist in the network? How do they relate to
        musical genres?}  
        Community detection (Louvain, Edge Betweenness), modularity, and genre purity.

  \item \textbf{What collaborative strategies do emerging artists adopt compared to
        established artists?}  
        Artist classification by popularity and collaborations; collaboration matrices
        between classes and centrality differences.

\end{enumerate}
\section{Datasets}
\label{datasets}

The initial dataset used for this analysis was downloaded from
\href{https://www.kaggle.com/datasets/jfreyberg/spotify-artist-feature-collaboration-network}{Spotify
  Artist Feature \& Collaboration Network} \cite{kaggle-dataset}. After careful
inspection, it was considered a reliable source, as it is derived from publicly
available data provided by the Spotify API and already pre-processed to
represent collaboration relationships among artists.

The dataset is structured as a undirected graph and consists of two main files:
\begin{itemize}
  \item \textbf{nodes.csv} -- contains the nodes of the graph, where each node represents an artist. The columns include:
        \begin{itemize}
          \item \texttt{id}: unique identifier of the artist.
          \item \texttt{name}: name of the artist.
          \item \texttt{followers}: number of followers of the artist on Spotify.
          \item \texttt{popularity}: popularity index (0--100), computed by Spotify based on recent streams and overall visibility.
          \item \texttt{genres}: list of genres associated with the artist.
          \item \texttt{chart\_hits}: List showing the number of Spotify chart hits in different countries.
        \end{itemize}
  \item \textbf{edges.csv} -- contains the edges of the graph, representing a collaboration between two artists. The columns include:
        \begin{itemize}
          \item \texttt{source}: ID of the collaborating artist.
          \item \texttt{target}: ID of the artist being collaborated with.
        \end{itemize}
\end{itemize}

The initial objective was to enrich the graph with additional artist-level
information, specifically \textbf{nationality} and \textbf{dominant musical
  genre}, in order to enable more in-depth social and cultural analyses of the
network.

\subsection{Artist Nationality Enrichment}
\label{subsec:nationality}

To associate a nationality with each artist, two complementary strategies were
adopted:

\begin{enumerate}
  \item \textbf{Inference based on musical genre.}
        In the first approach, nationality was inferred by analyzing the associated musical genres. For instance, an artist labeled with the genre \texttt{italian hip hop} was classified as \emph{Italian}. This method allowed the automatic assignment of nationality to a substantial subset of artists; however, it was not applicable in all cases, as many genres do not contain explicit geographical references.

  \item \textbf{Completion using an external dataset (MusicBrainz).}
        For artists whose nationality could not be inferred in the first step, data from the \texttt{MusicBrainz Dump (mbdump)} \cite{musicbrainz} were integrated. A direct matching based solely on artist names posed significant challenges due to the presence of homonyms with different nationalities. To mitigate this issue, the integration was performed exclusively on artists that remained unclassified after the first inference step, thereby improving overall precision and preserving data consistency.
\end{enumerate}

This hybrid procedure increased the coverage of nationality information and
enabled a more accurate subsequent analysis, particularly when comparing
artistic communities across different countries.

\subsection{Addition of Musical Genre}

To associate one or more musical genres with artists in the dataset, a
multi-level procedure was designed with the goal of maximizing coverage while
preserving semantic consistency. Each stage operates only on artists that
remained unclassified after the previous step.

\begin{itemize}
  \item \textbf{Direct mapping of Spotify genres.}
        Spotify-specific genres were first normalized and mapped to a limited set of musical macro-categories (e.g., \emph{Pop}, \emph{Rock}, \emph{Hip Hop / Rap}, \emph{Electronic / Dance}) using lexical and keyword-based rules.

  \item \textbf{AI-assisted completion.}
        Genres that could not be mapped automatically were classified through an AI-assisted process, which assigned them to the predefined macro-categories based on semantic similarity.

  \item \textbf{Inference via artistic collaborations.}
        For artists still lacking a genre, the collaboration network was exploited by assigning the most frequent genres among direct collaborators. When necessary, this inference was extended using a Breadth-First Search (BFS) up to three levels.

  \item \textbf{Popularity-based inference.}
        Remaining unclassified artists were analyzed using quantitative indicators such as follower count and popularity, inferring the most likely genres based on patterns observed in the dataset.

  \item \textbf{Global fallback assignment.}
        In the few remaining cases, a fallback strategy based on the most common genres in the dataset was applied.
\end{itemize}

This multi-level strategy enabled the creation of a dataset that is complete
and consistent from a musical genre perspective, minimizing arbitrary
assignments and integrally exploiting semantic, structural, and quantitative
information. The final result constitutes a solid foundation for the subsequent
network and artistic community analyses.

\section{Validity and Reliability}
\label{validity-reliability}
The dataset provides a realistic representation of musical collaborations on
Spotify, as it is derived from Spotify API data and models collaborations as
reciprocal relationships between artists. This abstraction is suitable for
capturing structural properties of the contemporary music collaboration
ecosystem.

Limitations arise from platform-specific biases and data incompleteness, since
not all collaborations or dimensions of artistic influence are observable on
Spotify. Additionally, inferred attributes such as nationality and genre may
introduce minor approximations, especially for artists with hybrid identities.
These effects are mitigated by conservative, multi-stage enrichment procedures
designed to reduce systematic bias.
Reliability is ensured through the use of public data sources and deterministic
preprocessing pipelines. All enrichment steps follow clearly defined and ordered
rules, allowing the analysis to be replicated with the same inputs.

The only partially non-deterministic component is the AI-assisted genre
classification; however, it is applied to a limited subset and its outputs are
fixed before analysis, preserving reproducibility of the results.


\section{Measures and Results}
\label{measures}

\subsection{General Analysis of the Italian Musical Collaboration Network}

In order to outline the key structural differences and obtain an overview of
the topology and internal dynamics of the Italian musical collaboration
network, a series of general metrics were calculated, as reported in Table
\ref{tab:metrics_italia}.

The \textbf{number of connected components} was measured to identify the
presence of isolated subgroups within the national musical ecosystem. Both the
\textbf{maximum degree} and the \textbf{average degree} of nodes were
calculated to assess the network's global connectivity and to quantify the
intensity of artists' collaborative activity. The network's \textbf{density}
was calculated to measure its overall cohesion. The \textbf{diameter} and the
\textbf{average shortest path length} provide an assessment of information flow
efficiency and indicate the ease with which artists can reach each other.
Finally, the \textbf{average clustering coefficient} and \textbf{transitivity}
were calculated to measure the network's local cohesion and the tendency
towards the formation of tightly-knit groups.

\begin{table}[h]
  \centering
  \caption{Structural characteristics of the Italian musical collaboration network}
  \label{tab:metrics_italia}
  \begin{tabular}{|l|c|}
    \hline
    \textbf{Parameter}             & \textbf{Value} \\
    \hline
    Total nodes (artists)          & 1,656          \\
    Total edges (collaborations)   & 4,307          \\
    Connected components           & 16             \\
    Maximum node degree            & 114            \\
    Average node degree            & 5.20           \\
    Density                        & 0.00314        \\
    Diameter                       & 10             \\
    Average shortest path length   & 4.14           \\
    Average clustering coefficient & 0.119          \\
    Transitivity                   & 0.128          \\
    \hline
  \end{tabular}
\end{table}

\subsubsection{Interpretation of Network Metrics}

The sixteen connected components suggest the existence of isolated groups,
corresponding to niches or artistic communities with limited contact with the
rest of the national ecosystem. Each artist is connected, on average, to about
five colleagues in the network. However, the degree distribution is highly
heterogeneous: the presence of a node with degree 114 reveals a central
\textit{hub} of major importance, while 54.6\% of artists have a degree of 1,
thus collaborating with only one other artist. The extremely low density
(approximately 0.31\% of possible connections are realized) confirms the
\textbf{sparse} nature of the network. The network diameter is 10, indicating
that any two artists can be connected through at most 10 intermediate
collaborations. Despite the low density, the network presents a modest average
path length (4.14), indicating that artists are connected through few
intermediate steps.

The values of the average clustering coefficient (0.119) and transitivity
(0.128) are moderate and close to each other. This indicates a measurable,
though not dominant, tendency towards \textbf{triadic closure}: two
collaborators of the same artist have approximately a 12\% probability of
having collaborated with each other in turn. This local cohesion fosters the
formation of cohesive artistic circles and partially clustered communities,
contributing to the stability of collaborative relationships and the sharing of
artistic practices within subgroups, while maintaining sufficient openness to
allow connections between different communities.

\subsubsection{Comparative Analysis with Major European Countries}

To situate the Italian results in a broader continental context, the analysis
was extended to major European countries. This approach enables a comparative
evaluation of collaborative dynamics.

Table \ref{tab:comparative_results_eu} shows the top five European countries by
number of artists (Italy, France, Germany, United Kingdom, and the Netherlands)
and, subsequently, other analyzed countries, providing a comprehensive picture
of the main European collaborative networks.

\begin{table}[h]
  \centering
  \caption{Comparison with major European countries}
  \label{tab:comparative_results_eu}
  \resizebox{\textwidth}{!}{%
    \begin{tabular}{|l|c|c|c|c|c|}
      \hline
      \textbf{Metric}      & \textbf{Italy} & \textbf{France} & \textbf{Germany} & \textbf{United Kingdom} & \textbf{Netherlands} \\
      \hline
      Total nodes          & 1,656 (3rd)    & 1,643 (4th)     & 2,706 (2nd)      & 3,290 (1st)             & 1,420 (5th)          \\
      Total edges          & 4,307 (5th)    & 4,754 (4th)     & 5,929 (2nd)      & 7,532 (1st)             & 5,143 (3rd)          \\
      Average degree       & 5.20 (5th)     & 5.79 (2nd)      & 4.38 (7th)       & 4.58 (6th)              & 7.24 (1st)           \\
      Connected components & 16 (4th)       & 27 (6th)        & 32 (7th)         & 70 (9th)                & 12 (3rd)             \\
      Average clustering   & 0.119 (4th)    & 0.113 (6th)     & 0.120 (3rd)      & 0.062 (9th)             & 0.151 (2nd)          \\
      \hline
    \end{tabular}%
  }
\end{table}

\paragraph{European Structural Models}

The analysis reveals significant structural differences between countries,
highlighting distinct models of musical collaboration:
\begin{enumerate}
  \item \textbf{Netherlands}: They have the highest \textbf{average degree} in Europe (7.24) despite ranking only fifth in number of artists. Their \textbf{average clustering coefficient} (0.151) is the second highest, and the percentage of hub artists (17.8\%) is the maximum in Europe.

  \item \textbf{Poland}: With an \textbf{average degree} of 6.20 and an \textbf{average clustering coefficient} of 0.167, they represent a model of a highly cohesive and interconnected network. Only 8 \textbf{connected components} indicate excellent structural integration.

  \item \textbf{Greece}: Has the highest \textbf{average clustering coefficient} in Europe (0.183) and a very good \textbf{average degree} (5.47). With only 4 \textbf{connected components}, it is one of the best-integrated networks.

  \item \textbf{United Kingdom}: Despite its larger size (3,290 artists), it presents the lowest \textbf{average clustering coefficient} (0.062) and the highest fragmentation (70 \textbf{connected components}), reflecting a vast but segmented market.
        It also shows a significantly lower network density
        (0.001392) compared to that observed for Italy (0.003143). This difference
        reflects two distinct structural configurations: on one hand, a British network
        that is extensive but characterized by high dispersion of collaborative
        relationships; on the other, an Italian ecosystem of more contained dimensions
        but relatively denser and more cohesive.
\end{enumerate}

\paragraph{Multidimensional Analysis and Interpretation of European Patterns}

\begin{figure}[htbp]
  \centering
  \includegraphics[width=0.95\textwidth]{network_size_eu.png}
  \caption{Comparative multidimensional analysis of musical collaboration networks at the European level.}
  \label{fig:network_analysis_eu}
\end{figure}

Figure \ref{fig:network_analysis_eu} presents a comparative multidimensional
analysis of musical collaboration networks at the European level. The analysis
synthesizes three critical aspects of the structure of European networks: size,
collaborative intensity, and the relationship between these variables. The
histograms present the top 15 European countries for two distinct metrics: on
the left, the number of artists; in the center, the total number of
collaborations. In the latter, we observe:

\begin{itemize}
  \item The \textbf{United Kingdom} (7,532 collaborations) and \textbf{Germany} (5,927
        collaborations) rank at the top for the overall number of collaborations; this
        result is primarily attributable to their high number of nodes.

  \item The \textbf{Netherlands} (5,143 collaborations) represent the most relevant
        case: despite being fifth in number of artists, they rank third for
        collaboration volume, surpassing both Italy (4,307) and France (4,754). This
        indicates intense collaborative activity.

  \item \textbf{Poland} (3,391 collaborations) shows surprising intensity, surpassing countries with larger networks such as Sweden (2,644) and Finland (2,329).
\end{itemize}

\subparagraph{Structural Relationship Between Artists and Collaborations}

The chart on the right explores the fundamental relationship between a
network's size (number of artists) and its activity (number of collaborations).
The relationship between the number of artists and the total number of
collaborations was analyzed using linear regression estimated by the least
squares method, considering all European countries included in the study. The
regression line can therefore be used as a reference: countries positioned
above it show a level of collaboration higher than expected given their size,
while those below exhibit more contained collaborative activity.

\begin{itemize}
  \item Countries positioned \textbf{above the regression line} (Netherlands, Poland,
        Greece) represent hyper-collaborative ecosystems, where the volume of
        interactions systematically exceeds expectations given their dimensions. These
        systems are typically characterized by high values of density and clustering
        coefficient.

  \item Countries \textbf{close to the line} (Italy, France, Germany, United Kingdom)
        follow an approximately linear relationship between collaborative activity and
        size.

  \item The \textbf{dispersion} of the data confirms the absence of a single European
        model, highlighting instead a plurality of structural configurations.
\end{itemize}

\subsubsection{Comparative Analysis with Major Non-European Countries}

To situate the Italian musical ecosystem in a broader global context, a
comparative analysis was conducted, extended to 49 non-European countries.
Table \ref{tab:comparative_results_global} presents the comparison between
Italy and the top 5 non-European countries by network size.

\begin{table}[h]
  \centering
  \caption{Comparison with the top 5 non-European countries by network size}
  \label{tab:comparative_results_global}
  \resizebox{\textwidth}{!}{%
    \begin{tabular}{|l|c|c|c|c|c|c|}
      \hline
      \textbf{Metric}      & \textbf{Italy} & \textbf{United States (1st)} & \textbf{Brazil (2nd)} & \textbf{India (3rd)} & \textbf{Japan (4th)} & \textbf{Mexico (5th)} \\
      \hline
      Total nodes          & 1,656          & 6,217                        & 1,859                 & 999                  & 890                  & 773                   \\
      Total edges          & 4,307          & 14,860                       & 5,547                 & 2,834                & 1,103                & 1,665                 \\
      Average degree       & 5.20           & 4.78                         & 5.97                  & 5.67                 & 2.48                 & 4.31                  \\
      Connected components & 16             & 111                          & 3                     & 2                    & 38                   & 4                     \\
      Average clustering   & 0.119          & 0.082                        & 0.173                 & 0.181                & 0.042                & 0.156                 \\
      Density              & 0.00314        & 0.000769                     & 0.003212              & 0.005685             & 0.002788             & 0.005580              \\
      \hline
    \end{tabular}%
  }
\end{table}

\paragraph{Global Dimensional Asymmetry}

The analysis reveals a marked dimensional asymmetry between the analyzed
musical systems:

\begin{itemize}
  \item The \textbf{United States} present exceptional dimensional characteristics,
        with a network of 6,217 artists and 14,860 collaborations, over 3.7 times
        larger than the Italian network. This considerable size is accompanied by a low
        density (0.000769) that reflects the large scale and diversification of the US
        music market, with 111 distinct groups operating in a relatively isolated
        manner. The average degree (4.78) is slightly lower than the Italian figure
        (5.20), indicating that, despite the larger size, the US network shows lower
        intensity of connections per artist.

  \item \textbf{Brazil} (1,859 artists) and \textbf{India} (999 artists) represent the major emerging musical systems, both characterized by high-density networks (0.003212 and 0.005685 respectively) and high structural cohesion. These values indicate highly integrated musical ecosystems, with only 3 and 2 distinct groups respectively, reflecting a strong cultural and geographical unity.

  \item Italy ranks sixth among non-European countries globally for size, surpassing
        countries such as South Korea (708) and Australia (653). Compared to Japan
        (890), Italy shows a much more cohesive structure: Japan, despite having more
        artists, presents a more fragmented network (38 distinct groups) and a lower
        average degree (2.48), while Italy, with 1,656 artists, maintains strong
        integration (16 components) and high collaborative activity (average degree
        5.20).
\end{itemize}

\paragraph{Global Collaborative Models}

The analysis identifies several models of musical ecosystems characterized by
intense collaborative activity:

\begin{itemize}
  \item \textbf{Puerto Rico} represents the most pronounced case of hyper-collaboration, with a particularly high average degree (10.66) and a high clustering coefficient (0.314).

  \item \textbf{Brazil} and \textbf{India} show similar structures, with high average degrees (5.97 and 5.67 respectively) and significant clustering (0.173 and 0.181). These values suggest vibrant ecosystems with a strong tendency towards the formation of cohesive communities, supported by low fragmentation (3 and 2 distinct groups), which indicates unified and well-integrated music scenes.

  \item \textbf{South Korea} presents an interesting case of a moderately large network (708 artists) but with high density (0.006185) and significant average degree (4.37), reflecting a structured and interconnected music scene.

  \item Italy, with an average degree of 5.20, ranks among the countries with the
        highest collaborative activity globally, surpassing countries such as the
        United States (4.78), Germany (4.38), the United Kingdom (4.58), and South
        Korea (4.37). This positioning indicates a musical culture particularly
        oriented towards collaboration, which compensates for the more contained
        network size with a greater intensity of interactions.
\end{itemize}

\subparagraph{Density and Fragmentation}
\begin{itemize}
  \item The analyzed \textbf{African} countries (Ghana, Egypt, Nigeria, Algeria)
        represent the model of small, ultra-dense networks, with densities ranging from
        0.047 to 0.143. These networks, despite having a limited number of artists
        (14-66 nodes), show extremely high internal interconnection. In particular,
        Ghana (0.138) and Algeria (0.143) present densities over 40 times higher than
        the US figure (0.000769), indicating extremely compact music scenes where
        almost all artists collaborate directly with each other.

  \item \textbf{Latin American} networks show intermediate densities but significantly higher than European and North American networks: Venezuela (0.077), Panama (0.065), Colombia (0.010), Argentina (0.010). These values, ranging from 3 to 10 times the Italian density, reflect regionally cohesive music scenes that are sufficiently broad to support a certain internal diversification. The low fragmentation (2-5 connected components) confirms the high degree of integration of these networks.

  \item Italy (density 0.00314) occupies an intermediate position in the global
        panorama:
        \begin{itemize}
          \item 4 times higher than that of the United States (0.000769)
          \item similar to that of Brazil (0.003212)
          \item about half that of India (0.005685) and Mexico (0.005580)
          \item higher than that of Japan (0.002788)
        \end{itemize}
        This positioning reflects a network that effectively balances size (1,656 nodes) and cohesion.
\end{itemize}

\subparagraph{Clustering and Local Cohesion}
\begin{itemize}
  \item The highest values of the \textbf{clustering coefficient} are observed in
        \textbf{Puerto Rico} (0.314), \textbf{Ghana} (0.287), and the \textbf{Dominican
          Republic} (0.275). 

  \item \textbf{India} (0.181) and \textbf{Brazil} (0.173) show high clustering values which, combined with the significant sizes of their networks (999 and 1,859 nodes respectively), reflect music scenes that combine breadth with strong internal cohesion. These values, approximately 1.5-2 times higher than the Italian one, indicate musical ecosystems in which the formation of closely interconnected artistic communities is noted.

  \item Italy (0.119) presents a moderate but significant clustering value in the
        global context. The comparison with major non-European countries reveals that:
        \begin{itemize}
          \item Italian clustering is higher than that of the United States (0.082) and Japan
                (0.042)
          \item It is lower than that of India (0.181), Brazil (0.173), and Mexico (0.156)
          \item It occupies an intermediate position among countries with networks of similar
                dimensions
        \end{itemize}
        This positioning indicates a network that balances local cohesion and openness: sufficiently cohesive to foster the formation of stable artistic circles, but also open enough to allow new connections and external exchanges.

  \item The low clustering values in countries such as \textbf{Canada} (0.037),
        \textbf{Australia} (0.035), and \textbf{China} (0.027) suggest network
        structures less inclined towards triadic closure. These values are
        approximately 3-4 times lower than the Italian one.
\end{itemize}
\subsection{Nodes}
\subsubsection{Centrality Measures}

To identify the structurally most important artists in the collaboration
network, four centrality measures were calculated. The objective is to
understand which artists occupy strategic positions and how these positions
manifest themselves through different aspects of the network's structure.

\paragraph{Degree Centrality}

\textbf{Degree centrality} quantifies the number of direct connections of a node, normalized by the maximum possible number of connections. In the analyzed musical network, this measure represents the number of distinct artists with whom a given artist has collaborated.

The implementation calculates both the absolute degree and the normalized
degree centrality. The results reveal a strongly asymmetric distribution: the
average degree centrality is 0.0031, while the maximum value reaches 0.0689.
This distribution indicates that the majority of artists maintain a limited
number of collaborations, while a small group of nodes concentrates a
significantly high number of connections.

Table \ref{tab:degree-centrality} presents the ten artists with the highest
degree centrality:
\begin{table}[h]
  \centering
  \caption{Top 10 artisti per Degree Centrality}
  \label{tab:degree-centrality}
  \begin{tabular}{c l r}
    \hline
    \textbf{Posizione} & \textbf{Artista} & \textbf{Degree Centrality} \\
    \hline
    1                  & Guè              & 0.0689                     \\
    2                  & Andrea Bocelli   & 0.0622                     \\
    3                  & Clementino       & 0.0508                     \\
    \hline
  \end{tabular}
\end{table}
Table~\ref{tab:degree-centrality} reports the ten artists with the highest degree centrality. Guè emerges as the most connected artist, with 114 distinct collaborations, confirming his central role in the Italian music network. Andrea Bocelli ranks second with 103 collaborations, highlighting his ability to connect across different musical genres. The strong presence of hip hop artists among the top positions reflects the collaborative nature of this genre and its structural relevance within the network. Ennio Morricone’s inclusion in the top ten introduces an element of stylistic and generational diversity.

\paragraph{Eigenvector Centrality}

\textbf{Eigenvector centrality} assigns importance not only to the quantity of connections but also to their quality: an artist has high eigenvector centrality if they are connected to other artists who themselves occupy central positions in the network. The algorithm converges iteratively, assigning each node a score proportional to the sum of the scores of its adjacent nodes.

The results show a highly concentrated distribution, with an average value of 0.0089 and a maximum of 0.2573, indicating the presence of a small and cohesive core of highly influential artists.

\begin{table}[h]
  \centering
  \caption{Top 10 artists by Eigenvector Centrality}
  \label{tab:eigenvector-centrality}
  \begin{tabular}{c l r}
    \hline
    \textbf{Rank} & \textbf{Artist} & \textbf{Eigenvector Centrality} \\
    \hline
    1             & Guè             & 0.2573                          \\
    2             & Gemitaiz        & 0.2069                          \\
    3             & Emis Killa      & 0.1904                          \\
    \hline
  \end{tabular}
\end{table}

Table~\ref{tab:eigenvector-centrality} presents the ten artists with the highest eigenvector centrality. Guè clearly dominates the ranking, with a value approximately 25% higher than the second-ranked artist, confirming his central role within the network. The ranking is exclusively composed of hip hop artists, revealing a genre-specific core that defines the structural center of the collaboration network. Artists such as Gemitaiz, Marracash, Jake La Furia, Don Joe, MadMan, and Lazza form a densely interconnected group. Notably, Andrea Bocelli is absent from this ranking despite his high degree centrality, suggesting that his collaborations are less embedded in the network’s central core.

\begin{figure}[H]
  \centering
  \includegraphics[width=0.5\linewidth]{images/eigenvector.pdf}
  \caption{Gephi representation of Eigenvector}
  \label{fig:placeholder}
\end{figure}

\paragraph{Closeness Centrality}
\textbf{Closeness centrality} measures how close a node is to all other nodes in the network by computing the inverse of the average shortest-path distance. In the context of musical collaborations, an artist with high closeness centrality can reach any other artist through a small number of intermediaries, indicating a structurally advantageous position for communication and interaction.

The distribution shows an average value of 0.2336 and a maximum of 0.3677, suggesting that even the most central artists require only a few steps to access the entire network.

\begin{table}[h]
  \centering
  \caption{Top 10 artists by Closeness Centrality}
  \label{tab:closeness-centrality}
  \begin{tabular}{c l r}
    \hline
    \textbf{Rank} & \textbf{Artist} & \textbf{Closeness Centrality} \\
    \hline
    1                  & Guè              & 0.3677                        \\
    2                  & Clementino       & 0.3576                        \\
    3                  & Gemitaiz         & 0.3537                        \\
    \hline
  \end{tabular}
\end{table}

Table~\ref{tab:closeness-centrality} reports the ten artists with the highest closeness centrality. Guè again occupies the top position, confirming his optimal placement within the network. Artists such as Clementino and J-AX stand out more in this metric than in eigenvector centrality, indicating a bridging role between different network regions rather than membership in the densest core. The presence of Elisa and Rocco Hunt further highlights the ability of this network to support efficient connectivity across genres. Overall, closeness centrality identifies artists who facilitate rapid information flow and collaboration opportunities across the entire musical ecosystem.

\paragraph{Betweenness Centrality}

\textbf{Betweenness centrality} measures how often a node lies on the shortest paths between other pairs of nodes. In a collaboration network, artists with high betweenness act as intermediaries connecting different clusters, even if they do not have many direct collaborations. For computational efficiency, the metric was approximated by sampling 1,000 nodes.

The distribution is highly skewed, with an average of 0.0018 and a maximum of 0.1250, indicating that only a few artists control the flow of connections between different network communities. Andrea Bocelli dominates this metric, highlighting his role as a structural bridge between classical, pop, and other genres. Artists such as Ennio Morricone, DJ Matrix, Jovanotti, and Cristina D’Avena further confirm that betweenness centrality captures a bridging role distinct from local centrality. Clementino combines high centrality in multiple metrics, linking the hip hop core with other parts of the network. Table~\ref{tab:betweenness-centrality} lists the ten artists with the highest betweenness centrality.

\begin{table}[h]
  \centering
  \caption{Top 10 artists by Betweenness Centrality}
  \label{tab:betweenness-centrality}
  \begin{tabular}{c l r}
    \hline
    \textbf{Rank} & \textbf{Artist} & \textbf{Betweenness Centrality} \\
    \hline
    1                  & Andrea Bocelli   & 0.1250                          \\
    2                  & Clementino       & 0.0910                          \\
    3                  & Guè              & 0.0642                          \\
    \hline
  \end{tabular}
\end{table}

\subsubsection{Assortativity}

Assortativity analysis examines connection tendencies based on specific node
attributes. Three forms of assortativity were calculated to characterize
collaboration patterns in the Italian musical network.

\paragraph{Degree Assortativity}

The degree assortativity coefficient of the network is \textbf{-0.1052}, indicating a slightly \textbf{disassortative} structure. This negative value shows that highly connected artists tend to collaborate with less connected artists, rather than forming tight clusters among themselves.

In the musical network, this pattern suggests that hubs do not form an isolated core but include less prolific artists in their collaborations. This may reflect several dynamics: established artists providing visibility to emerging talent, producers and featuring artists working across a wide spectrum of collaborators, or a generally open music scene where past collaborations do not limit future opportunities. Although moderate, the negative assortativity supports a structure that promotes mobility and access for less central artists, contrasting with rigid hierarchical networks where only the most successful figures collaborate among themselves.

\paragraph{Followers Assortativity}

The followers assortativity coefficient is \textbf{0.0724}, a positive but near-zero value, indicating a very weak tendency towards assortative patterns based on popularity. This metric measures whether artists with a similar number of Spotify followers preferentially collaborate with each other.

The result suggests that popularity has limited influence on collaboration patterns. Highly followed artists do not exclusively collaborate with similarly popular artists, nor do less popular artists remain isolated. Categorizing edges into High-High, Low-Low, and High-Low collaborations using the median follower count shows a relatively balanced distribution, with both intra- and inter-popularity collaborations occurring.

This indicates a fluid Italian music scene, where fame does not strongly constrain partnerships. While the slightly positive coefficient hints at a weak homophilic tendency—very popular artists collaborating somewhat more frequently with other popular artists—it is not strong enough to create significant structural barriers within the network.

\paragraph{Genre Assortativity and Modularity}

The musical genre assortativity coefficient is \textbf{0.4778}, a clearly positive value indicating a strong tendency for artists to collaborate within the same or related genres. The modularity is \textbf{Q = 0.2987}, moderate according to the interpretative scale, showing that while genre strongly influences collaborations, cross-genre connections still maintain global network cohesion.

\begin{table}[h]
  \centering
  \caption{Top 10 genre pairs in collaborations}
  \label{tab:genre-pairs}
  \begin{tabular}{c l r l}
    \hline
    \textbf{Rank} & \textbf{Genre 1}           & \textbf{N. collab.} & \textbf{Genre 2} \\
    \hline
    1             & Hip Hop / Rap               & 1829                & (intra-genere)    \\
    2             & Pop                         & 819                 & (intra-genere)    \\
    3             & Hip Hop / Rap               & 596                 & Pop               \\
    4             & Hip Hop / Rap               & 104                 & Indie             \\
    5             & Elettronica / Dance         & 85                  & (intra-genere)    \\
    6             & Classica / Orchestrale      & 72                  & (intra-genere)    \\
    7             & Elettronica / Dance         & 62                  & Pop               \\
    8             & Indie                       & 45                  & Pop               \\
    9             & Soundtrack / Film Score     & 37                  & (intra-genere)    \\
    10            & Elettronica / Dance         & 34                  & Hip Hop / Rap     \\
    \hline
  \end{tabular}
\end{table}

Table~\ref{tab:genre-pairs} lists the ten most frequent genre pairs. Hip Hop / Rap dominates with 1829 intra-genre collaborations, followed by Pop with 819. The Hip Hop / Rap + Pop combination is the most frequent cross-genre connection (596 edges), acting as a structural bridge between the two major genres. Other notable cross-genre collaborations include Hip Hop / Rap + Indie and Elettronica / Dance + Pop, though less frequent. Smaller yet cohesive communities are found in Classica / Orchestrale, Elettronica / Dance, and Soundtrack / Film Score.

Overall, the results reveal a pattern of genre specialization with selective cross-genre interactions. While artists tend to collaborate predominantly within their stylistic boundaries, established bridges, especially between hip hop and pop, ensure network connectivity. Hip Hop / Rap’s high internal density confirms its central role and highly interconnected internal ecosystem.

\subsection{Network}
\subsubsection{Community Detection}
Community detection analysis was conducted to investigate whether artists tend
to collaborate primarily with other artists belonging to the same musical
macro-genre. To this end, two distinct approaches were applied, namely the
Louvain algorithm and the Edge Betweenness (Girvan--Newman) method. For each
approach, both the number of detected communities and their genre homogeneity
were evaluated by measuring the purity of the dominant macro-genre within each
community.

\noindent
\subsubsection*{Louvain}

\noindent
\begin{minipage}[t]{0.55\textwidth}
  \setlength{\parindent}{0pt}
  The Louvain algorithm identified a total of 34 communities, revealing a
  relatively fragmented network structure. Several communities exhibit a high
  degree of genre homogeneity, particularly for the \emph{Hip Hop / Rap} and
  \emph{Pop} macro-genres, with purity values exceeding 0.6 and reaching 1.0 in
  smaller clusters. At the same time, many communities show a mixed composition,
  with multiple dominant macro-genres coexisting. This behavior is visually
  reflected in the community layout produced by the Louvain algorithm in Gephi
  (Figure~\ref{fig:louvain}), where dense, genre-centered clusters coexist with
  more diffuse, heterogeneous structures. Overall, this indicates that artist
  collaborations are not strictly constrained by genre boundaries, especially
  within larger communities where cross-genre interactions are more frequent.
\end{minipage}
\hfill
\begin{minipage}[t]{0.40\textwidth}
  \centering
  \vspace{0pt}
  \includegraphics[width=\linewidth]{images/louvain.pdf}
  \captionof{figure}{Community structure obtained by applying the Louvain
    algorithm in Gephi.}
  \label{fig:louvain}
\end{minipage}

\subsubsection*{Louvain versus Genre Assortativity}
The community detection results can be interpreted using genre assortativity
and modularity measures. The genre assortativity coefficient
($r = 0.4778$) indicates a clear homophilic tendency, with artists more likely
to collaborate within the same or closely related macro-genres. This behavior
is reflected in the Louvain partition, which identifies high-purity communities,
particularly for \emph{Hip Hop / Rap} and \emph{Pop}.

The moderate modularity value ($Q = 0.2987$) suggests that genre does not
induce a strong structural separation of the network. Several communities
exhibit a mixed genre composition, especially in the Edge Betweenness
partition, where genre purity is often below $0.5$. This pattern is explained
by the presence of numerous cross-genre collaborations, notably between
\emph{Hip Hop / Rap} and \emph{Pop}, which act as structural bridges and limit
overall modularity.

Overall, the network displays a pattern of \emph{selective mixing}: local
genre-based homophily coexists with cross-genre ties, resulting in a structure
that is both cohesive and interconnected.

\subsubsection*{Edge Betweenness}
The Edge Betweenness (Girvan--Newman) algorithm produced 17 communities, resulting in a coarser partitioning of the network compared to the Louvain method. The identified communities are generally less pure, with genre purity values frequently below 0.5, particularly in larger clusters dominated by \emph{Pop} and \emph{Hip Hop / Rap}. This outcome indicates that the iterative removal of highly central edges tends to group together artists from different macro-genres, emphasizing the presence of bridge nodes and inter-genre collaborations rather than a clear separation based on musical genre.

\subsubsection{Degree Distribution}

\noindent
\begin{minipage}[t]{0.55\textwidth}
  \setlength{\parindent}{0pt}
  The analysis of the degree distribution provides insight into the global
  structure of the artist collaboration network. The minimum degree of 1 reflects
  the presence of artists involved in a single collaboration, while the maximum
  degree of 114 highlights a small set of highly connected nodes acting as hubs.
  The average degree of 5.20 indicates an overall sparse network.

  This heterogeneous connectivity pattern is clearly visible in the Gephi
  visualization (Figure~\ref{fig:degree_gephi}), where node size is proportional
  to degree. A large number of small nodes coexist with a few prominent hubs,
  suggesting a strongly right-skewed distribution. This structure is typical of
  complex networks, in which highly connected nodes play a central role in
  maintaining global connectivity and facilitating interactions across different
  regions of the network, potentially spanning multiple musical genres.
\end{minipage}
\hfill
\begin{minipage}[t]{0.40\textwidth}
  \centering
  \vspace{0pt}
  \includegraphics[width=\linewidth]{images/degree.pdf}
  \captionof{figure}{Artist collaboration network visualized in Gephi with node
    size proportional to degree.}
  \label{fig:degree_gephi}
\end{minipage}

\vspace{1em}

\begin{figure}[H]
  \centering
  \subfloat[Degree distribution in linear scale.\label{fig:dd_linear}]{
    \includegraphics[width=0.48\textwidth]{images/DD-AC.png}
  }
  \hfill
  \subfloat[Degree distribution in logarithmic scale.\label{fig:dd_log}]{
    \includegraphics[width=0.48\textwidth]{images/DDLG.png}
  }
  \caption{Degree distribution of the artist collaboration network.}
  \label{fig:degree_distribution}
\end{figure}

\section{Conclusion}
\label{conclusion}
The investigation demonstrates that the graph of Italian musical collaborations on
Spotify exhibits structural properties typical of real-world complex networks: high sparsity,
heterogeneous degree distribution with a long tail, and the presence of central hubs.
The quantitative analysis of centrality metrics reveals a significant concentration of
relational power within a cohesive core dominated by the Hip Hop/Rap genre, while betweenness
measures identify strategic bridge nodes connecting distinct stylistic communities. Assortativity
and modularity coefficients indicate strong homophily based on genre, yet with sufficient
permeability to ensure global interconnectedness. The transnational comparative analysis,
based on linear regression between node and edge cardinality, places Italy above the average European trend,
characterizing it as a case of a hyper-collaborative, non-scalable ecosystem.
The results confirm the utility of network metrics for dynamically modeling the production structure
of a cultural sector.

Qualitative analysis of the quantitative findings of the study.

\section{Critique}
\label{critique}

Despite the analysis having produced significant and interpretable results, it
is important to acknowledge the main intrinsic limitations of our
methodological approach and the data used.

\begin{itemize}[leftmargin=*]

  \item \textbf{Partial and platform-limited data}: The dataset includes exclusively collaborations officially registered on Spotify, excluding those occurring on other platforms (YouTube, SoundCloud), in live contexts, or in unofficial forms. This may lead to an underestimation of network density, especially for underground genres or emerging artists operating outside mainstream channels.

  \item \textbf{Automatic attribute inference}: Nationality and musical genre were assigned through automated procedures. While this maximized coverage, it may have introduced systematic errors or excessive simplifications.

  \item \textbf{Static versus dynamic analysis}: The network was treated as a fixed snapshot in time. A longitudinal approach would have allowed studying how communities form, how artists' centrality changes in response to events (album release, chart entry), and how collaborative strategies evolve across different career stages.

  \item \textbf{Simplicity of the network model}: The graph is unimodal (only artists) and undirected, and does not distinguish between occasional collaborations and stable partnerships. A richer representation, perhaps weighted according to the number of shared tracks or enriched with temporal metadata, would have enabled more accurate analysis.

  \item \textbf{Interpretive limits of the structural approach}: Quantitative analysis identifies connection patterns; one can measure that an artist connects two musical communities, but it is not known whether this is a conscious creative choice, a personal relationship, or a commercial strategy. Without integrating structural data with qualitative sources, such as interviews, lyric analysis, reconstruction of production contexts, or market dynamics, conclusions about "collaborative strategies" or the "social role" of artists remain plausible yet not overly detailed hypotheses.
\end{itemize}

\begin{thebibliography}{9}

  \bibitem{project-repo}
  \textit{Network Analysis Project Repository},
  GitHub,
  \url{https://github.com/edefbo1/Network_Analysis.git}

  \bibitem{gephi}
  Gephi Consortium,
  \textit{Gephi: an open source graph visualization and analysis software},
  Gephi.org,
  \url{https://gephi.org/}

  \bibitem{networkx}
  \textit{NetworkX Documentation and Tutorial},
  NetworkX.org,
  \url{https://networkx.org/documentation/stable/tutorial.html}

  \bibitem{kaggle-dataset}
  J. Freyberg,
  \textit{Spotify Artist Feature Collaboration Network},
  Kaggle Dataset,
  \url{https://www.kaggle.com/datasets/jfreyberg/spotify-artist-feature-collaboration-network/}

  \bibitem{musicbrainz}
  MusicBrainz Foundation,
  \textit{MusicBrainz Database — Download and Documentation},
  MusicBrainz.org,
  \url{https://musicbrainz.org/doc/MusicBrainz_Database/Download}

\end{thebibliography}

\end{document}