\PassOptionsToPackage{unicode}{hyperref}
\PassOptionsToPackage{hyphens}{url}
%
\documentclass[12pt, a4paper]{article}
\usepackage[a4paper,margin=1in]{geometry}
\setlength\parindent{0pt}
\usepackage{mathptmx}
\usepackage{amsmath,amssymb}
\usepackage[T1]{fontenc}
\usepackage[utf8]{inputenc}
\usepackage{textcomp}
\usepackage{graphicx}
\usepackage{hyperref}
\usepackage{float}
\usepackage{enumitem}
\usepackage{subfig}
\usepackage{titlesec}

\setcounter{secnumdepth}{4} 
\titleformat{\paragraph}
  {\normalfont\normalsize\bfseries}{\theparagraph}{1em}{}
\renewcommand{\theparagraph}{\thesubsubsection.\arabic{paragraph}}

\begin{document}
\maketitle

\section{Introduction}
\label{introduction}

L'industria musicale ha subito trasformazioni radicali negli ultimi due
decenni, passando da un modello tradizionale basato sulla vendita fisica di
dischi a un ecosistema digitale dominato dalle piattaforme di streaming.
Spotify, lanciato nel 2008, è diventato il principale servizio di streaming
musicale a livello globale, con oltre 500 milioni di utenti attivi e un
catalogo che supera i 100 milioni di brani.

In questo nuovo panorama, le collaborazioni musicali, comunemente note come
"featuring", hanno assunto un ruolo centrale. Se in passato le collaborazioni
erano eventi relativamente rari e spesso limitati a progetti speciali, oggi
rappresentano una strategia fondamentale per artisti di ogni livello di
popolarità. Le collaborazioni permettono agli artisti di raggiungere nuove
audience, sperimentare con generi musicali diversi, aumentare la propria
visibilità sulle piattaforme di streaming e creare connessioni strategiche
all'interno dell'industria musicale.

La Social Network Analysis (SNA) offre strumenti metodologici potenti per
studiare queste dinamiche relazionali. Rappresentando gli artisti come nodi e
le loro collaborazioni come archi, è possibile costruire una rete che cattura
la complessità delle interazioni nel panorama musicale contemporaneo.

Questo studio si concentra sull'analisi delle collaborazioni tra artisti su
Spotify, con un focus particolare sulla scena musicale italiana come caso di
studio principale.

\subsection*{Domande di Ricerca}
Attraverso una prospettiva di rete, questo lavoro mira a rispondere a domande
fondamentali sul funzionamento dell'ecosistema musicale italiano e il suo
posizionamento nel contesto internazionale:
\begin{enumerate}[leftmargin=*, itemsep=5pt]
  \item Quali sono le caratteristiche strutturali fondamentali della rete di
        collaborazioni musicali italiane su Spotify?
  \item Come si posiziona la scena italiana rispetto ai principali paesi europei ed non
        europei?
  \item Chi sono gli artisti più centrali e influenti, e che tipo di ruolo strutturale
        ricoprono?
  \item Come sono distribuiti i collegamenti nella rete?
  \item Quali sono i pattern di collaborazione prevalenti? Gli artisti tendono a
        collaborare con colleghi simili per grado di connessione, popolarità o genere
        musicale?
  \item Esistono comunità ben definite nella rete? Come si relazionano con i generi
        musicali?
  \item Quali strategie collaborative adottano gli artisti emergenti rispetto agli
        artisti affermati?
\end{enumerate}

Per rispondere a questi interrogativi, lo studio applica un'ampia gamma di
metriche e tecniche di analisi di rete a un dataset ricavato da Spotify,
opportunamente arricchito con informazioni su nazionalità e genere degli
artisti.

\section{Datasets}
\label{datasets}

The initial dataset used for this analysis was downloaded from
\href{https://www.kaggle.com/datasets/jfreyberg/spotify-artist-feature-collaboration-network}{Spotify
  Artist Feature \& Collaboration Network} \cite{kaggle-dataset}. After careful
inspection, it was considered a reliable source, as it is derived from publicly
available data provided by the Spotify API and already pre-processed to
represent collaboration relationships among artists.

The dataset is structured as a directed graph and consists of two main files:
\begin{itemize}
  \item \textbf{nodes.csv} -- contains the nodes of the graph, where each node represents an artist. The columns include:
        \begin{itemize}
          \item \texttt{id}: unique identifier of the artist.
          \item \texttt{name}: name of the artist.
          \item \texttt{followers}: number of followers of the artist on Spotify.
          \item \texttt{popularity}: popularity index (0--100), computed by Spotify based on recent streams and overall visibility.
          \item \texttt{genres}: list of genres associated with the artist.
          \item \texttt{chart\_hits}: List showing the number of Spotify chart hits in different countries (according to kworb.net)
        \end{itemize}
  \item \textbf{edges.csv} -- contains the edges of the graph, representing a collaboration between two artists. The columns include:
        \begin{itemize}
          \item \texttt{source}: ID of the collaborating artist.
          \item \texttt{target}: ID of the artist being collaborated with.
        \end{itemize}
\end{itemize}

The initial objective was to enrich the graph with additional artist-level
information, specifically \textbf{nationality} and \textbf{dominant musical
  genre}, in order to enable more in-depth social and cultural analyses of the
network.

\subsection{Artist Nationality Enrichment}
\label{subsec:nationality}

To associate a nationality with each artist, two complementary strategies were
adopted:

\begin{enumerate}
  \item \textbf{Inference based on musical genre.}
        In the first approach, nationality was inferred by analyzing the associated musical genres. For instance, an artist labeled with the genre \texttt{italian hip hop} was classified as \emph{Italian}. This method allowed the automatic assignment of nationality to a substantial subset of artists; however, it was not applicable in all cases, as many genres do not contain explicit geographical references.

  \item \textbf{Completion using an external dataset (MusicBrainz).}
        For artists whose nationality could not be inferred in the first step, data from the \texttt{MusicBrainz Dump (mbdump)} \cite{musicbrainz} were integrated. A direct matching based solely on artist names posed significant challenges due to the presence of homonyms with different nationalities. To mitigate this issue, the integration was performed exclusively on artists that remained unclassified after the first inference step, thereby improving overall precision and preserving data consistency.
\end{enumerate}

This hybrid procedure increased the coverage of nationality information and
enabled a more accurate subsequent analysis, particularly when comparing
artistic communities across different countries.

\subsection{Aggiunta del genere artistico}
Per associare uno o più generi musicali agli artisti del dataset, è stata
progettata una procedura con l’obiettivo di massimizzare il numero di artisti
con genere musicale assegnato, mantenendo al contempo coerenza e plausibilità
semantica delle assegnazioni. Il processo si articola in più fasi successive,
ciascuna delle quali interviene solo sugli artisti rimasti privi di genere nel
passaggio precedente.

\begin{itemize}
  \item \textbf{Mappatura diretta dei generi Spotify.}
        In una prima fase, i generi specifici forniti da Spotify (ad esempio \emph{italian pop}, \emph{alternative rock}, \emph{deep house}) sono stati normalizzati e ricondotti a un insieme limitato di \emph{macro-categorie} musicali (come \emph{Pop}, \emph{Rock}, \emph{Hip Hop / Rap}, \emph{Elettronica / Dance}, ecc.). Questa mappatura è stata realizzata tramite regole lessicali e keyword-based, consentendo di ridurre l’elevata frammentazione dei generi originali e di ottenere una rappresentazione più compatta e comparabile.

  \item \textbf{Completamento assistito tramite AI.}
        I generi che non risultavano mappabili automaticamente nella fase precedente (raccolti nella categoria \emph{Altri / Specifici}) sono stati estratti e forniti come input a un processo di classificazione assistito da intelligenza artificiale. L’AI ha ricondotto ciascun genere residuo a una delle macro-categorie musicali precedentemente definite, sulla base di similarità semantiche e conoscenza musicale generale. Le associazioni così ottenute sono state successivamente reintegrate nel dataset, consentendo di ridurre in modo significativo il numero di generi non classificati e di migliorare la copertura complessiva della mappatura, mantenendo coerenza con lo schema di categorizzazione adottato.

  \item \textbf{Inferenza tramite collaborazioni artistiche.}
        Per gli artisti privi di genere dopo la mappatura diretta, è stato sfruttato il grafo delle collaborazioni. In particolare, sono stati assegnati i generi più frequenti tra i collaboratori diretti; qualora ciò non fosse sufficiente, l’inferenza è stata estesa tramite una ricerca BFS (Breadth-First Search) fino a tre livelli di distanza nella rete, selezionando i generi più ricorrenti nei nodi visitati.

  \item \textbf{Inferenza basata su metriche di popolarità.}
        Gli artisti ancora non classificati sono stati analizzati in base a indicatori quantitativi come numero di follower e popolarità. Attraverso semplici euristiche derivate da pattern osservati nel dataset (ad esempio alta popolarità associata a generi mainstream), sono stati inferiti i generi più probabili.

  \item \textbf{Assegnazione di fallback globale.}
        Infine, per i rari casi rimasti senza genere, è stato applicato un meccanismo di fallback basato sui generi globalmente più comuni nel dataset, garantendo che ogni artista fosse associato ad almeno una macro-categoria musicale.

\end{itemize}

Questa strategia multilivello ha permesso di ottenere un dataset completo e
consistente dal punto di vista dei generi musicali, riducendo al minimo le
assegnazioni arbitrarie e sfruttando in modo integrato informazioni semantiche,
strutturali e quantitative. Il risultato finale costituisce una base solida per
le successive analisi di rete e di comunità artistiche.

\section{Validity and Reliability}
\label{validity-and-reliability-not-needed-for-the-project-proposal}

The initial dataset is derived from the Spotify API and pre-processed to
explicitly model collaboration relationships among artists, providing a
reasonable approximation of actual musical interactions on the platform.
However, the dataset reflects Spotify’s ecosystem and temporal snapshot, and
therefore may not capture collaborations occurring outside the platform or
informal artistic relationships.

Additional artist-level attributes, such as nationality and dominant musical
genre, were introduced to support higher-level social and cultural analyses.
Nationality was inferred using a hybrid approach combining genre-based cues and
external data from MusicBrainz, applied selectively to reduce ambiguity due to
artist name homonyms. While this procedure increases coverage and interpretive
power, it introduces a degree of uncertainty, particularly for artists whose
identity or geographic origin is weakly signaled by available metadata.

Similarly, musical genres were consolidated into a limited set of
macro-categories through a multi-step process involving rule-based mapping,
AI-assisted classification, and network-based inference. This approach improves
comparability and completeness, but necessarily abstracts away finer-grained
genre distinctions and may propagate local biases through collaboration-based
inference.

Regarding reliability, all data processing and enrichment steps follow
deterministic rules or documented heuristics, ensuring that the analysis is
reproducible given the same inputs and parameters. External data sources and
AI-assisted mappings represent potential sources of variability; however, their
use was constrained to well-defined stages and applied consistently across the
dataset. Overall, the adopted methodology yields a dataset that is both valid
and reliable for social network analysis, as it is grounded in authoritative
data sources, enriched through controlled and well-documented procedures, and
constructed to balance descriptive accuracy with reproducibility, making it a
solid foundation for the subsequent network and community analyses.

\section{Misure e Risultati}
\label{measures}
In questa sezione si riassumono in modo sintetico le principali misure utilizzate, le tecnologie impiegate e il loro legame con gli obiettivi dello studio.
\subsection*{Rappresentazione della rete}
\begin{itemize}
  \item Grafo non orientato $G = (V, E)$: nodi = artisti (\texttt{spotify\_id}), archi
        = collaborazioni tra artisti presenti nelle tracce.
  \item Implementazione in Python con \texttt{pandas} per i CSV dei nodi/archi e
        \texttt{NetworkX} per la costruzione del grafo e il calcolo delle misure.
\end{itemize}

\subsection*{Misure di centralità}
\begin{itemize}
  \item \textbf{Degree centrality}: normalizza il numero di collaborazioni di ciascun artista, identifica gli hub più connessi e viene usata per selezionare i top artisti nel sottografo di analisi.
  \item \textbf{Betweenness centrality}: misura quante volte un artista cade sui cammini minimi tra coppie di nodi, individuando i ``broker'' strutturali tra comunità e generi diversi.
  \item \textbf{Closeness centrality}: inverso della distanza media da un artista a tutti gli altri, quantifica quanto rapidamente un artista può raggiungere il resto della rete.
  \item \textbf{Eigenvector centrality}: assegna punteggi più alti agli artisti collegati ad altri artisti centrali, catturando l’appartenenza al ``core'' della scena.
\end{itemize}

\subsection*{Community detection e bridge}
\begin{itemize}
  \item \textbf{Louvain}: individua comunità massimizzando la modularità, permettendo di associare cluster strutturali a macro-generi, scene nazionali o gruppi di etichetta.
  \item \textbf{Edge betweenness} e \textbf{constraint} di Burt: identificano rispettivamente collaborazioni-ponte tra comunità e artisti con accesso a \textit{structural holes}, fondamentali per la diffusione di stili e contenuti tra mondi diversi.
\end{itemize}

\subsection*{Generi, nazionalità e successo}
\begin{itemize}
  \item Generi e nazionalità sono gestiti come attributi dei nodi (\texttt{genre},
        \texttt{nationality}); si contano collaborazioni intra/inter-genere e
        intra/inter-nazionali per valutare assortatività e aperture transnazionali.
  \item Per gli artisti senza genere, il genere viene inferito dal genere più frequente
        nel vicinato di rete, con soglia minima di collaborazioni per garantire
        robustezza.
  \item Le misure strutturali sono correlate con indicatori esterni
        (\texttt{popularity} Spotify, numero collaborazioni, collaborazioni estere,
        presenza in chart) per studiare il legame tra posizione nella rete, popolarità
        ed espansione internazionale.
\end{itemize}

\subsection*{Artisti emergenti}
\begin{itemize}
  \item Si costruisce un \texttt{DataFrame} con \texttt{popularity} e numero di
        collaborazioni per artista; soglie su entrambi gli indicatori definiscono tre
        classi: \emph{emergente}, \emph{intermedio}, \emph{affermato}.
  \item La matrice delle collaborazioni tra classi (emergente–emergente,
        emergente–affermato, ecc.) mostra le strategie di networking (orizzontale tra
        pari vs collegamento verso artisti affermati) e come queste si riflettano nella
        crescita di centralità e popolarità.
\end{itemize}
\subsection{Analisi generale della rete di collaborazioni musicali italiane}

Al fine di delineare le differenze strutturali chiave e ottenere una panoramica
della topologia e delle dinamiche interne della rete di collaborazioni musicali
italiane, è stata calcolata una serie di misure generali, come riportato nella
Tabella \ref{tab:metrics_italia}.

Il \textbf{numero di componenti connesse} è stato misurato per identificare la
presenza di sottogruppi isolati all'interno dell'ecosistema musicale nazionale.
Sia il \textbf{grado massimo} che il \textbf{grado medio} dei nodi sono stati
calcolati per valutare la connettività globale della rete e per quantificare
l'intensità dell'attività collaborativa degli artisti. La \textbf{densità}
della rete è stata calcolata per misurare la sua coesione complessiva. Il
\textbf{diametro} e la \textbf{lunghezza media del percorso più breve}
forniscono una valutazione dell'efficienza del flusso di informazioni e
indicano la facilità con cui gli artisti possono essere raggiunti gli uni
attraverso gli altri. Infine, il \textbf{coefficiente di clustering medio} e la
\textbf{transitività} sono stati calcolati per misurare la coesione locale
della rete e la tendenza alla formazione di gruppi strettamente connessi.

I sedici componenti connessi suggeriscono l'esistenza di gruppi isolati,
corrispondenti a nicchie o comunità artistiche con contatti limitati con il
resto dell'ecosistema nazionale. Ciascun artista è connesso, in media, a circa
cinque colleghi nella rete. Tuttavia, la distribuzione del grado è fortemente
eterogenea: la presenza di un nodo con grado 114 rivela un \textit{hub} di
importanza centrale, mentre il 54.6\% degli artisti presenta un grado pari a 1,
collaborando dunque con un solo altro artista. La densità estremamente bassa
(circa lo 0.31\% delle possibili connessioni è realizzata) conferma la natura
\textbf{sparsa} della rete. Il diametro della rete è pari a 10, indicando che
due artisti qualsiasi possono essere collegati attraverso al massimo 10
collaborazioni intermedie. Nonostante la bassa densità, la rete presenta una
lunghezza media del percorso contenuta (4.14), indicando che gli artisti sono
collegati attraverso poche collaborazioni intermedie.

I valori del coefficiente di clustering medio (0.119) e della transitività
(0.128) sono moderati e tra loro vicini. Ciò indica una tendenza misurabile,
seppur non dominante, alla \textbf{chiusura triadica}: due collaboratori di uno
stesso artista hanno una probabilità di circa il 12\% di aver collaborato a
loro volta. Questa coesione locale favorisce la formazione di circoli artistici
coesi e comunità parzialmente clusterizzate, contribuendo alla stabilità delle
relazioni collaborative e alla condivisione di pratiche artistiche all'interno
di sottogruppi, pur mantenendo sufficiente apertura per permettere connessioni
tra comunità diverse.

\subsection{Analisi comparativa con i principali paesi europei}

Per collocare i risultati italiani in un contesto continentale più ampio,
l’analisi è stata estesa ai principali paesi europei. Questo approccio consente
una valutazione comparativa delle dinamiche collaborative.

La Tabella \ref{tab:comparative_results_eu} mostra i primi cinque paesi europei
per numero di artisti (Italia, Francia, Germania, Regno Unito e Paesi Bassi) e,
successivamente, altri paesi analizzati, fornendo un quadro completo delle
principali reti collaborative europee.


Dall'analisi emergono differenze strutturali significative tra i paesi,
evidenziando modelli distinti di collaborazione musicale:
\begin{enumerate}
  \item \textbf{Paesi Bassi}: Presentano l'\textbf{average degree} più alto d'Europa (7.24) nonostante siano solo quinti per numero di artisti. Il loro \textbf{average clustering coefficient} (0.151) è il secondo più alto, e la percentuale di hub artists (17.8\%) è la massima in Europa.

  \item \textbf{Polonia}: Con un \textbf{average degree} di 6.20 e \textbf{average clustering coefficient} di 0.167, rappresentano un modello di rete altamente coesa e interconnessa. Solo 8 \textbf{connected components} indicano un'eccellente integrazione strutturale.

  \item \textbf{Grecia}: Ha il \textbf{average clustering coefficient} più alto d'Europa (0.183) e un ottimo \textbf{average degree} (5.47). Con soli 4 \textbf{connected components}, è una delle reti meglio integrate.

  \item \textbf{Regno Unito}: Nonostante le dimensioni maggiori (3.290 artisti), presenta il \textbf{average clustering coefficient} più basso (0.062) e la frammentazione più alta (70 \textbf{connected components}), riflettendo un mercato vasto ma segmentato.
        Presenta anche una densità di rete significativamente più bassa
        (0.001392) rispetto a quella osservata per l'Italia (0.003143). Tale differenza
        riflette due configurazioni strutturali distinte: da un lato, una rete
        britannica estesa ma caratterizzata da un'elevata dispersione delle relazioni
        collaborative; dall'altro, un ecosistema italiano di dimensioni più contenute
        ma relativamente più denso e coeso.

\end{enumerate}

\subsubsection{Analisi multidimensionale e interpretazione dei pattern europei}

\begin{figure}[htbp]
  \centering
  \includegraphics[width=0.95\textwidth]{network_size_eu.png}
  \caption{Analisi multidimensionale comparativa delle reti di collaborazione musicale a livello europeo.}
  \label{fig:network_analysis_eu}
\end{figure}
La Figura \ref{fig:network_analysis_eu} presenta un'analisi multidimensionale comparativa
delle reti di collaborazione musicale a livello europeo.
L'analisi sintetizza tre aspetti critici della struttura
delle reti europee: dimensione, intensità collaborativa e relazione tra queste
variabili.
Gli istogrammi presentano i primi 15 paesi europei per due metriche distinte: a
sinistra il numero di artisti, al centro il numero totale di collaborazioni. In
quest'ultimo si nota:

\begin{itemize}
  \item Il \textbf{Regno Unito} (7.532 collaborazioni) e la \textbf{Germania} (5.927)
        si collocano ai vertici per numero complessivo di collaborazioni; tale
        risultato è principalmente riconducibile alla loro elevata numerosità di nodi.

  \item I \textbf{Paesi Bassi} (5.143 collaborazioni) rappresentano il caso più
        rilevante: pur essendo quinti per numero di artisti, si posizionano terzi per
        volume di collaborazioni, superando sia l'Italia (4.307) che la Francia
        (4.754). Ciò indica un'intensa attività collaborativa.

  \item La \textbf{Polonia} (3.391 collaborazioni) mostra un'intensità sorprendente,
        superando paesi con reti più estese come Svezia (2.644) e Finlandia (2.329).
\end{itemize}

\subsubsection{Relazione Strutturale tra Artisti e Collaborazioni}

Il grafico a destra esplora la relazione fondamentale tra la dimensione di una
rete (numero di artisti) e la sua attività (numero di collaborazioni). La
relazione tra il numero di artisti e il numero totale di collaborazioni è stata
analizzata mediante una regressione lineare stimata con il metodo dei minimi
quadrati, considerando tutti i paesi europei inclusi nello studio. La retta
ottenuta, \( y = 2.44x - 76 \), descrive l'andamento medio delle reti musicali
europee.

\begin{itemize}
  \item \textbf{Coefficiente angolare (2.44)}: indica che, in media, a ogni
        artista aggiuntivo corrispondono circa 2.44 collaborazioni in più.
        Questo valore esprime l'intensità collaborativa media dei network
        musicali analizzati.

  \item \textbf{Intercetta (-76)}: il valore negativo suggerisce che reti molto
        piccole tendono ad avere un numero di collaborazioni inferiore a quello
        previsto dalla tendenza generale. In particolare, al di sotto di circa
        30 artisti, la struttura della rete risulta meno attiva.
\end{itemize}

La retta di regressione può quindi essere utilizzata come riferimento: i paesi
posizionati al di sopra mostrano un livello di collaborazione superiore a
quanto atteso rispetto alle loro dimensioni, mentre quelli al di sotto
presentano un'attività collaborativa più contenuta.

\begin{itemize}
  \item I paesi che si collocano \textbf{sopra la retta di regressione} (Paesi Bassi,
        Polonia, Grecia) rappresentano ecosistemi iper-collaborativi, dove il volume di
        interazioni supera sistematicamente le aspettative date le dimensioni. Questi
        sistemi sono tipicamente caratterizzati da alti valori di densità e
        coefficiente di clustering.

  \item I paesi \textbf{prossimi alla retta} (Italia, Francia, Germania, Regno Unito)
        seguono una relazione approssimativamente lineare, dell'attività collaborativa
        in funzione della dimensione.

  \item La \textbf{dispersione} dei dati conferma l'assenza di un modello unico
        europeo, evidenziando invece una pluralità di configurazioni strutturali.
\end{itemize}

\subsection{Analisi comparativa con i principali paesi non-europei}

Per collocare l'ecosistema musicale italiano in un contesto globale più ampio,
è stata condotta un'analisi comparativa estesa a 49 paesi non-europei.

La Tabella \ref{tab:comparative_results_global} presenta il confronto tra
l'Italia e i primi 5 paesi non-europei per dimensione della rete.


L'analisi rivela una marcata asimmetria dimensionale tra i sistemi musicali
analizzati:

\begin{itemize}
  \item Gli \textbf{Stati Uniti} presentano caratteristiche dimensionali eccezionali,
        con una rete di 6.217 artisti e 14.860 collaborazioni, oltre 3,7 volte più
        grande della rete italiana. Questa dimensione considerevole è accompagnata da
        una bassa densità (0.000769) che riflette l'ampia scala e diversificazione del
        mercato musicale statunitense, con ben 111 gruppi distinti che operano in modo
        relativamente isolato. Il grado medio (4.78) è leggermente inferiore a quello
        italiano (5.20), indicando che, nonostante le dimensioni maggiori, la rete
        statunitense mostra una minore intensità di connessioni per artista.

  \item Il \textbf{Brasile} (1.859 artisti) e l'\textbf{India} (999 artisti)
        rappresentano i maggiori sistemi musicali emergenti, entrambi caratterizzati da
        reti ad alta densità (0.003212 e 0.005685 rispettivamente) e da un'elevata
        coesione strutturale. Questi valori indicano ecosistemi musicali altamente
        integrati, con soli 3 e 2 gruppi distinti rispettivamente, che riflettono una
        forte unità culturale e geografica.

  \item L'Italia si posiziona al sesto posto a livello globale non-europeo per
        dimensione, superando paesi come Corea del Sud (708) e Australia (653).
        Rispetto al Giappone (890), l'Italia mostra una struttura molto più coesa: il
        Giappone, pur avendo più artisti, presenta una rete più frammentata (38 gruppi
        distinti) e un grado medio inferiore (2.48), mentre l'Italia, con 1.656
        artisti, mantiene una forte integrazione (16 componenti) e un'elevata attività
        collaborativa (grado medio 5.20).
\end{itemize}

L'analisi identifica diversi modelli di ecosistemi musicali caratterizzati da
intensa attività collaborativa:

\begin{itemize}
  \item \textbf{Porto Rico} rappresenta il caso più marcato di iper-collaboratività, con un grado medio particolarmente elevato (10.66) e un coefficiente di clustering alto (0.314).

  \item Il \textbf{Brasile} e l'\textbf{India} mostrano strutture simili, con gradi
        medi elevati (5.97 e 5.67 rispettivamente) e clustering significativo (0.173 e
        0.181). Questi valori suggeriscono ecosistemi vibranti con forte tendenza alla
        formazione di comunità coese, supportata da una bassa frammentazione (3 e 2
        gruppi distinti) che indica scene musicali unificate e ben integrate.

  \item La \textbf{Corea del Sud} presenta un caso interessante di rete moderatamente
        grande (708 artisti) ma con alta densità (0.006185) e grado medio significativo
        (4.37), riflettendo una scena musicale strutturata e interconnessa.

  \item L'Italia, con un grado medio di 5.20, si colloca tra i paesi con maggiore
        attività collaborativa a livello globale, superando paesi come Stati Uniti
        (4.78), Germania (4.38), Regno Unito (4.58) e Corea del Sud (4.37). Questo
        posizionamento indica una cultura musicale particolarmente orientata alla
        collaborazione, che compensa le dimensioni più contenute della rete con una
        maggiore intensità di interazioni.
\end{itemize}

\subsubsection{Densità e frammentazione}
\begin{itemize}
  \item I paesi \textbf{africani} analizzati (Ghana, Egitto, Nigeria, Algeria)
        rappresentano il modello delle reti piccole ultra-dense, con densità che vanno
        da 0.047 a 0.143. Queste reti, pur avendo un numero limitato di artisti (14-66
        nodi), mostrano un'elevatissima interconnessione interna. In particolare, Ghana
        (0.138) e Algeria (0.143) presentano densità oltre 40 volte superiori a quella
        statunitense (0.000769), indicando scene musicali estremamente compatte dove
        quasi tutti gli artisti collaborano direttamente tra loro.

  \item Le reti \textbf{latinoamericane} mostrano densità intermedie ma
        significativamente superiori alle reti europee e nordamericane: Venezuela
        (0.077), Panama (0.065), Colombia (0.010), Argentina (0.010). Questi valori,
        compresi tra 3 e 10 volte la densità italiana, riflettono scene musicali
        regionalmente coese ma sufficientemente ampie da sostenere una certa
        diversificazione interna. La bassa frammentazione (2-5 componenti connesse)
        conferma l'alto grado di integrazione di queste reti.

  \item L'Italia (densità 0.00314) si colloca in una posizione intermedia nel panorama
        globale. Se confrontata con i primi 5 paesi non-europei per dimensione,
        presenta una densità:
        \begin{itemize}
          \item 4 volte superiore a quella degli Stati Uniti (0.000769)
          \item simile a quella del Brasile (0.003212)
          \item circa la metà di quella dell'India (0.005685) e del Messico (0.005580)
          \item superiore a quella del Giappone (0.002788)
        \end{itemize}
        Questo posizionamento riflette una rete che bilancia efficacemente dimensione (1.656 nodi) e coesione.

  \item Tra i paesi con reti di dimensioni simili all'Italia (1.000-2.000 nodi),
        l'India (999 nodi, densità 0.005685) e la Corea del Sud (708 nodi, densità
        0.006185) mostrano densità quasi doppie rispetto all'Italia, riflettendo
        modelli di collaborazione più intensi. Tuttavia, l'Italia compensa con un grado
        medio (5.20) superiore a quello di molti di questi paesi, indicando che pur con
        una densità complessiva moderata, gli artisti italiani tendono a collaborare
        con un numero maggiore di colleghi.
\end{itemize}

\subsubsection{Clustering e coesione Locale}
\begin{itemize}
  \item I più alti valori di \textbf{clustering coefficient} si osservano in
        \textbf{Porto Rico} (0.314), \textbf{Ghana} (0.287) e \textbf{Repubblica
          Dominicana} (0.275). Questi valori eccezionalmente elevati, compresi tra 8 e 11
        volte quelli di paesi come Canada o Australia, indicano una forte tendenza alla
        formazione di "triadi chiuse" nelle quali i collaboratori di un artista tendono
        a collaborare frequentemente anche tra loro. Questo modello suggerisce comunità
        musicali estremamente coese.

  \item \textbf{India} (0.181) e \textbf{Brasile} (0.173) mostrano valori di clustering elevati che,
        combinati con le dimensioni significative delle loro reti (999 e 1.859 nodi rispettivamente),
        riflettono scene musicali che uniscono ampiezza a forte coesione interna. Questi valori,
        circa 1.5-2 volte superiori a quello italiano, indicano ecosistemi musicali in cui si
        nota la formazione di comunità artistiche
        strettamente interconnesse.

  \item L'Italia (0.119) presenta un valore di clustering moderato ma significativo nel
        contesto globale. Il confronto con i principali paesi non-europei rivela che:
        \begin{itemize}
          \item Il clustering italiano è superiore a quello degli Stati Uniti (0.082) e del
                Giappone (0.042)
          \item È inferiore a quello di India (0.181), Brasile (0.173) e Messico (0.156)
          \item Si colloca in una posizione intermedia tra i paesi con reti di dimensioni
                simili
        \end{itemize}
        Questo posizionamento indica una rete che bilancia coesione locale e apertura: sufficientemente
        coesa da favorire la formazione di circoli artistici stabili, ma anche abbastanza aperta da
        permettere nuove connessioni e scambi esterni.

  \item I bassi valori di clustering in paesi come \textbf{Canada} (0.037),
        \textbf{Australia} (0.035) e \textbf{Cina} (0.027) suggeriscono strutture
        reticolari meno inclini alla chiusura triadica. Questi valori, circa 3-4 volte
        inferiori a quello italiano.
\end{itemize}
\subsection{Nodes}
\subsubsection{Centrality Measures}

To identify the structurally most important artists in the collaboration network, four centrality measures were calculated. The objective is to understand which artists occupy strategic positions and how these positions manifest themselves through different aspects of the network's structure.

\paragraph{Degree Centrality}

\textbf{Degree centrality} quantifies the number of direct connections of a node, normalized by the maximum possible number of connections. In the analyzed musical network, this measure represents the number of distinct artists with whom a given artist has collaborated.

The implementation calculates both the absolute degree and the normalized degree centrality. The results reveal a strongly asymmetric distribution: the average degree centrality is 0.0031, while the maximum value reaches 0.0689. This distribution indicates that the majority of artists maintain a limited number of collaborations, while a small group of nodes concentrates a significantly high number of connections.

Table \ref{tab:degree-centrality} presents the ten artists with the highest degree centrality:

Guè emerges as the most connected node in the network with 114 distinct collaborations, establishing himself as a central hub of the Italian music scene. The presence of Andrea Bocelli in second place with 103 collaborations is particularly significant: despite working in a substantially different musical genre (classical/pop crossover), he has developed an extensive network of collaborations spanning multiple genres. The predominance of hip hop artists in the top positions (Clementino, Gemitaiz, Night Skinny, Don Joe, Inoki, Fabri Fibra, Emis Killa) confirms that this genre has a high propensity for collaboration and constitutes a central element in the network's structure. The presence of Ennio Morricone in tenth place introduces an element of generational and stylistic diversity into the ranking.

\paragraph{Eigenvector Centrality}

\textbf{Eigenvector centrality} assigns importance not only to the quantity of connections but also to their quality: an artist has high eigenvector centrality if they are connected to other artists who themselves occupy central positions in the network. The algorithm converges iteratively, assigning each node a score proportional to the sum of the scores of its adjacent nodes.

The results show an even more pronounced concentration compared to degree centrality: the average value is 0.0089 while the maximum reaches 0.2573, highlighting that a small number of artists form a highly cohesive central core.

Table \ref{tab:eigenvector-centrality} presents the ten artists with the highest eigenvector centrality:

Guè maintains the dominant position with a value considerably higher than the other artists (approximately 25\% more than the second-ranked artist), indicating that his collaborations predominantly involve the most central artists in the scene. The exclusive presence of hip hop artists in this ranking reveals the existence of a central core dominated by this genre. Artists such as Marracash, Jake La Furia, Don Joe, MadMan, and Lazza form a highly interconnected core that defines the center of the Italian hip hop network. It is noteworthy that Andrea Bocelli, despite having a high number of collaborations (second in degree centrality), is absent from this ranking, suggesting that his collaborations predominantly involve less central or more peripheral artists compared to the network's main core.

\begin{figure}
  \centering
  \includegraphics[width=0.5\linewidth]{images/eigenvector.pdf}
  \caption{Caption}
  \label{fig:placeholder}
\end{figure}

\paragraph{Closeness Centrality}
\textbf{Closeness centrality} measures the closeness of a node to all other nodes in the network, calculating the inverse of the average distance based on shortest paths. An artist with high closeness centrality can rapidly reach any other artist in the network through a limited number of intermediaries.

The results show an average value of 0.2336 with a maximum of 0.3677, indicating that even the most central artists require an average of about three steps to reach any other node in the network.

In the practical context of the music scene, high closeness centrality confers concrete strategic advantages: optimally positioned artists can more easily access information and emerging trends in different parts of the scene, initiate collaborations with distant artists in the network through few intermediaries, and rapidly disseminate their own style or musical innovations throughout the entire ecosystem. This metric therefore identifies artists who, regardless of their belonging to the central core, occupy positions that facilitate communication and the circulation of opportunities throughout the entire network.

Table \ref{tab:closeness-centrality} presents the ten artists with the highest closeness centrality:

Guè confirms his dominant position, being optimally positioned to access the entire network with the minimum number of intermediaries. Clementino and J-AX emerge particularly in this metric compared to eigenvector centrality, suggesting they occupy strategic bridging positions: although not in the densest core, their collaborations traverse different network clusters, allowing them to act as efficient connectors between different subcommunities. The presence of Elisa (ninth position) and Rocco Hunt (tenth position) introduces significant genre diversification. These artists, although not belonging to the dominant hip hop core, maintain positions of global closeness that facilitate the transmission of cross-genre influences. The network, although dominated by hip hop in its central core, therefore maintains a structure that allows efficient communication between different genres, suggesting a musical ecosystem where barriers between genre communities do not hinder the rapid circulation of ideas and collaborative opportunities.

\paragraph{Betweenness Centrality}

\textbf{Betweenness centrality} identifies nodes that are frequently found on the shortest paths between other pairs of nodes. Artists with high betweenness act as intermediaries between different components of the network, even without necessarily having a high number of direct connections. The calculation was approximated by sampling 1000 random nodes for computational efficiency reasons.

The results show a strongly asymmetric distribution: the average value is 0.0018 while the maximum reaches 0.1250, indicating that a very limited number of nodes controls the flow of connections between different clusters in the network.

Andrea Bocelli dominates this metric with a value significantly higher than the other artists (almost 40\% higher than the second-ranked artist), revealing his structural role as a bridge between different musical contexts. Despite not belonging to the hip hop core identified by eigenvector centrality, Bocelli connects classical and pop music with other genres, acting as an essential intermediary in the overall network structure. The presence of Ennio Morricone (fifth position), DJ Matrix (sixth position), Jovanotti (ninth position) and Cristina D'Avena (tenth position) - artists working in genres other than hip hop - confirms that betweenness centrality captures a structural role distinct from local centrality. These artists do not belong to the central core but occupy gatekeeping positions between different genre communities. Clementino maintains a high position in all analyzed metrics (third in degree, second in closeness and betweenness), confirming himself as one of the structurally most relevant artists in the network: an integral part of the hip hop core but with strong bridging capabilities towards other music scenes.

Table \ref{tab:betweenness-centrality} presents the ten artists with the highest betweenness centrality:

\subsubsection{Assortativity}

Assortativity analysis examines connection tendencies based on specific node attributes. Three forms of assortativity were calculated to characterize collaboration patterns in the Italian musical network.

\paragraph{Degree Assortativity}

The obtained degree assortativity coefficient is \textbf{-0.1052}, indicating a slightly \textbf{disassortative} network. This negative value means that artists with numerous collaborations tend to connect with artists who have fewer, rather than collaborating predominantly with each other.

In the context of the analyzed musical network, this pattern suggests that hubs (artists with high degree) do not form an isolated group but include less prolific artists in their collaborations. This may reflect various dynamics: established artists providing visibility to emerging artists, producers and featuring artists collaborating with a broad spectrum of artists at different activity levels, or more generally a relatively open music scene where the number of past collaborations does not constitute a significant barrier for future opportunities. The negative value, albeit moderate, is consistent with a structure that facilitates mobility and access even for less central artists, in contrast with rigidly hierarchical models where exclusively the most successful figures collaborate with each other.

\paragraph{Followers Assortativity}

The followers assortativity coefficient is \textbf{0.0724}, a positive but near-zero value, indicating a very weak tendency towards assortative patterns based on popularity.

This result suggests that popularity, measured through the number of Spotify followers, exerts a limited influence on collaboration patterns. Artists with a high number of followers do not show a marked preference for exclusive collaborations with other artists of similar popularity, nor does a clear segregation emerge between different popularity strata. Detailed analysis categorizes edges into High-High, Low-Low and High-Low using the median as a threshold, revealing that there are both collaborations between artists of similar popularity and collaborations with artists of a different popularity level from their own, in relatively balanced proportions.

This pattern indicates a relatively fluid Italian musical ecosystem regarding fame: collaborations are not strongly constrained by the number of followers, and established artists show willingness to collaborate with less-known colleagues. However, the slightly positive value suggests the existence of a weak homophilic preference: very popular artists collaborate more frequently with other popular artists, but this tendency does not create significant structural barriers.

\paragraph{Genre Assortativity and Modularity}

The results regarding musical genre show markedly different patterns compared to the other forms of assortativity. The genre assortativity coefficient is \textbf{0.4778}, a decidedly positive value indicating a strong assortative tendency. The modularity is \textbf{Q = 0.2987}, classifiable as moderate according to the implemented interpretative scale (between 0.1 and 0.3).

These values reveal that musical genre exerts a significant influence on collaboration patterns: artists tend to collaborate predominantly with other artists of the same genre or related genres. However, the moderate (not high) modularity indicates that this tendency does not produce impermeable compartments: there are numerous cross-genre collaborations that keep the network globally interconnected.

Table \ref{tab:genre-pairs} presents the ten most frequent genre pairs in collaborations:

Hip Hop / Rap dominates with 1829 intra-genre collaborations, confirming the high density and cohesion of this musical community. Pop presents 819 internal collaborations, also demonstrating significant internal cohesion. The Hip Hop / Rap + Pop combination accounts for 596 edges, constituting by far the most frequent cross-genre contamination. This reveals a fundamental structural bridge between the two dominant genres of the Italian scene. The Hip Hop / Rap + Indie (104 edges) and Elettronica / Dance + Pop (62 edges) combinations represent other significant contaminations, although with considerably lower frequency. Genres such as Classica/Orchestrale (72 internal edges), Elettronica/Dance (85 edges), and Soundtrack (37 edges) highlight communities of more limited size but still internally cohesive.

The ratio between intra-genre and inter-genre edges, calculated by the function through the analysis of overlaps between genre lists, shows that collaborations remain predominantly within genre boundaries, with relevant exceptions.

In the context of the Italian music scene, these results describe a picture of specialization with selective contamination: genres maintain distinct identities and artists collaborate predominantly within their stylistic boundaries, but there are consolidated bridges between complementary genres, particularly between hip hop and pop. The high density of hip hop (1829 internal collaborations) confirms what emerged from the centrality analysis: this genre not only dominates the central core of the network but has developed an extremely rich and interconnected internal ecosystem.

\subsection{Network}
\subsubsection{Community Detection}
Community detection analysis was conducted to investigate whether artists tend
to collaborate primarily with other artists belonging to the same musical
macro-genre. To this end, two distinct approaches were applied, namely the
Louvain algorithm and the Edge Betweenness (Girvan--Newman) method. For each
approach, both the number of detected communities and their genre homogeneity
were evaluated by measuring the purity of the dominant macro-genre within each
community.

\noindent
\subsubsection*{Louvain}

\noindent
\begin{minipage}[t]{0.55\textwidth}
  \setlength{\parindent}{0pt}
  The Louvain algorithm identified a total of 34 communities, revealing a
  relatively fragmented network structure. Several communities exhibit a high
  degree of genre homogeneity, particularly for the \emph{Hip Hop / Rap} and
  \emph{Pop} macro-genres, with purity values exceeding 0.6 and reaching 1.0 in
  smaller clusters. At the same time, many communities show a mixed composition,
  with multiple dominant macro-genres coexisting. This behavior is visually
  reflected in the community layout produced by the Louvain algorithm in Gephi
  (Figure~\ref{fig:louvain}), where dense, genre-centered clusters coexist with
  more diffuse, heterogeneous structures. Overall, this indicates that artist
  collaborations are not strictly constrained by genre boundaries, especially
  within larger communities where cross-genre interactions are more frequent.
\end{minipage}
\hfill
\begin{minipage}[t]{0.40\textwidth}
  \centering
  \vspace{0pt}
  \includegraphics[width=\linewidth]{images/louvain.pdf}
  \captionof{figure}{Community structure obtained by applying the Louvain
    algorithm in Gephi.}
  \label{fig:louvain}
\end{minipage}

\subsubsection*{Louvain versus Genre Assortativity}
The community detection results can be interpreted using genre assortativity
and modularity measures. The genre assortativity coefficient
($r = 0.4778$) indicates a clear homophilic tendency, with artists more likely
to collaborate within the same or closely related macro-genres. This behavior
is reflected in the Louvain partition, which identifies high-purity communities,
particularly for \emph{Hip Hop / Rap} and \emph{Pop}.

The moderate modularity value ($Q = 0.2987$) suggests that genre does not
induce a strong structural separation of the network. Several communities
exhibit a mixed genre composition, especially in the Edge Betweenness
partition, where genre purity is often below $0.5$. This pattern is explained
by the presence of numerous cross-genre collaborations, notably between
\emph{Hip Hop / Rap} and \emph{Pop}, which act as structural bridges and limit
overall modularity.

Overall, the network displays a pattern of \emph{selective mixing}: local
genre-based homophily coexists with cross-genre ties, resulting in a structure
that is both cohesive and interconnected.

\subsubsection*{Edge Betweenness}
The Edge Betweenness (Girvan--Newman) algorithm produced 17 communities, resulting in a coarser partitioning of the network compared to the Louvain method. The identified communities are generally less pure, with genre purity values frequently below 0.5, particularly in larger clusters dominated by \emph{Pop} and \emph{Hip Hop / Rap}. This outcome indicates that the iterative removal of highly central edges tends to group together artists from different macro-genres, emphasizing the presence of bridge nodes and inter-genre collaborations rather than a clear separation based on musical genre.

\subsubsection{Degree Distribution}

\noindent
\begin{minipage}[t]{0.55\textwidth}
  \setlength{\parindent}{0pt}
  The analysis of the degree distribution provides insight into the global
  structure of the artist collaboration network. The minimum degree of 1 reflects
  the presence of artists involved in a single collaboration, while the maximum
  degree of 114 highlights a small set of highly connected nodes acting as hubs.
  The average degree of 5.20 indicates an overall sparse network.

  This heterogeneous connectivity pattern is clearly visible in the Gephi
  visualization (Figure~\ref{fig:degree_gephi}), where node size is proportional
  to degree. A large number of small nodes coexist with a few prominent hubs,
  suggesting a strongly right-skewed distribution. This structure is typical of
  complex networks, in which highly connected nodes play a central role in
  maintaining global connectivity and facilitating interactions across different
  regions of the network, potentially spanning multiple musical genres.
\end{minipage}
\hfill
\begin{minipage}[t]{0.40\textwidth}
  \centering
  \vspace{0pt}
  \includegraphics[width=\linewidth]{images/degree.pdf}
  \captionof{figure}{Artist collaboration network visualized in Gephi with node
    size proportional to degree.}
  \label{fig:degree_gephi}
\end{minipage}

\vspace{1em}

\begin{figure}[H]
  \centering
  \subfloat[Degree distribution in linear scale.\label{fig:dd_linear}]{
    \includegraphics[width=0.48\textwidth]{images/DD-AC.png}
  }
  \hfill
  \subfloat[Degree distribution in logarithmic scale.\label{fig:dd_log}]{
    \includegraphics[width=0.48\textwidth]{images/DDLG.png}
  }
  \caption{Degree distribution of the artist collaboration network.
    The linear-scale histogram highlights the high concentration of low-degree
    nodes, while the logarithmic-scale representation emphasizes the long-tailed
    behavior induced by a small number of highly connected artists.}
  \label{fig:degree_distribution}
\end{figure}

\section{Conclusion}
\label{conclusion}
The investigation demonstrates that the graph of Italian musical collaborations on Spotify exhibits structural properties typical of real-world complex networks: high sparsity, heterogeneous degree distribution with a long tail, and the presence of central hubs. The quantitative analysis of centrality metrics reveals a significant concentration of relational power within a cohesive core dominated by the Hip Hop/Rap genre, while betweenness measures identify strategic bridge nodes connecting distinct stylistic communities. Assortativity and modularity coefficients indicate strong homophily based on genre, yet with sufficient permeability to ensure global interconnectedness. The transnational comparative analysis, based on linear regression between node and edge cardinality, places Italy above the average European trend, characterizing it as a case of a hyper-collaborative, non-scalable ecosystem. The results confirm the utility of network metrics for dynamically modeling the production structure of a cultural sector.

Qualitative analysis of the quantitative findings of the study.

\section{Critique}
\label{critique}

Nonostante l'analisi abbia prodotto risultati significativi e interpretabili, è
importante riconoscere le principali limitazioni intrinseche al nostro
approccio metodologico e ai dati utilizzati.

\begin{itemize}[leftmargin=*]

  \item \textbf{Dati parziali e limitati alla piattaforma}: Il dataset include esclusivamente collaborazioni registrate ufficialmente su Spotify, escludendo quelle avvenute su altre piattaforme (YouTube, SoundCloud), in contesti live, o in forme non ufficiali. Ciò può portare a una sottostima della densità della rete, specialmente per generi underground o per artisti emergenti che operano al di fuori dei canali mainstream.
  \item \textbf{Inferenza automatica degli attributi}: La nazionalità e il genere musicale sono stati assegnati tramite procedure automatizzate. Sebbene questo abbia massimizzato la copertura, può aver introdotto errori sistematici o semplificazioni eccessive.

  \item \textbf{Analisi statica vs. dinamica}: La rete è stata trattata come un’istantanea fissa nel tempo. Un approccio longitudinale avrebbe permesso di studiare come le comunità si formano, come la centralità degli artisti cambia in risposta a eventi (uscita di un album, ingresso in classifica), e come le strategie collaborative si evolvono nelle diverse fasi della carriera.

  \item \textbf{Semplicità del modello di rete}: Il grafo è monomodale (solo artisti) e non orientato, e non distingue tra collaborazioni occasionali e partnership stabili. Una rappresentazione più ricca, magari pesata in base al numero di tracce condivise o arricchita con metadati temporali, avrebbe permesso analisi più accurata.

  \item \textbf{Limiti interpretativi dell'approccio strutturale}: L'analisi quantitativa identifica pattern di connessione, si può misurare che un artista collega due comunità musicali, ma non sappiamo se si tratti di una scelta creativa consapevole, di una relazione personale, di una strategia commerciale. Senza integrare i dati strutturali con fonti qualitative, come interviste, analisi dei testi, ricostruzione dei contesti produttivi o dinamiche di mercato, le conclusioni sulle “strategie collaborative” o sul “ruolo sociale” degli artisti restano ipotesi plausibili ma non troppo dettagliate.
\end{itemize}

\begin{thebibliography}{9}

  \bibitem{project-repo}
  \textit{Network Analysis Project Repository},
  GitHub,
  \url{https://github.com/edefbo1/Network_Analysis.git}

  \bibitem{gephi}
  Gephi Consortium,
  \textit{Gephi: an open source graph visualization and analysis software},
  Gephi.org,
  \url{https://gephi.org/}

  \bibitem{networkx}
  \textit{NetworkX Documentation and Tutorial},
  NetworkX.org,
  \url{https://networkx.org/documentation/stable/tutorial.html}

  \bibitem{kaggle-dataset}
  J. Freyberg,
  \textit{Spotify Artist Feature Collaboration Network},
  Kaggle Dataset,
  \url{https://www.kaggle.com/datasets/jfreyberg/spotify-artist-feature-collaboration-network/}

  \bibitem{musicbrainz}
  MusicBrainz Foundation,
  \textit{MusicBrainz Database — Download and Documentation},
  MusicBrainz.org,
  \url{https://musicbrainz.org/doc/MusicBrainz_Database/Download}

\end{thebibliography}

\end{document}