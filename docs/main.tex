\PassOptionsToPackage{unicode}{hyperref}
\PassOptionsToPackage{hyphens}{url}
%
\documentclass[12pt, a4paper]{article}
\usepackage[a4paper,margin=1in]{geometry}
\setlength\parindent{0pt}
\usepackage{mathptmx}
\usepackage{amsmath,amssymb}
\usepackage[T1]{fontenc}
\usepackage[utf8]{inputenc}
\usepackage{textcomp}
\usepackage{graphicx}
\usepackage{hyperref}
\usepackage{float}
\usepackage{enumitem}
\usepackage{subfig}
\usepackage{titlesec}

\setcounter{secnumdepth}{4} 
\titleformat{\paragraph}
  {\normalfont\normalsize\bfseries}{\theparagraph}{1em}{}
\renewcommand{\theparagraph}{\thesubsubsection.\arabic{paragraph}}

\author{Federico , matricola 
\\Federica Santisi, matricola
\\Giorgia Pirelli, matricola}
\date{Dicembre 2025}
\title{Analisi delle Collaborazioni Musicali su Spotify: Una Prospettiva di Social Network Analysis}

\begin{document}
\maketitle

\section{Introduction}
\label{introduction}

L'industria musicale ha subito trasformazioni radicali negli ultimi due
decenni, passando da un modello tradizionale basato sulla vendita fisica di
dischi a un ecosistema digitale dominato dalle piattaforme di streaming.
Spotify, lanciato nel 2008, è diventato il principale servizio di streaming
musicale a livello globale, con oltre 500 milioni di utenti attivi e un
catalogo che supera i 100 milioni di brani.

In questo nuovo panorama, le collaborazioni musicali, comunemente note come
"featuring", hanno assunto un ruolo centrale. Se in passato le collaborazioni
erano eventi relativamente rari e spesso limitati a progetti speciali, oggi
rappresentano una strategia fondamentale per artisti di ogni livello di
popolarità. Le collaborazioni permettono agli artisti di raggiungere nuove
audience, sperimentare con generi musicali diversi, aumentare la propria
visibilità sulle piattaforme di streaming e creare connessioni strategiche
all'interno dell'industria musicale.

La Social Network Analysis (SNA) offre strumenti metodologici potenti per
studiare queste dinamiche relazionali. Rappresentando gli artisti come nodi e
le loro collaborazioni come archi, è possibile costruire una rete che cattura
la complessità delle interazioni nel panorama musicale contemporaneo.
Attraverso metriche di centralità, analisi di comunità e studio dei pattern di
connessione, possiamo identificare quali artisti occupano posizioni
strategiche, come si formano le comunità musicali e quali fattori influenzano
il successo delle collaborazioni.

Questo studio si concentra sull'analisi delle collaborazioni tra artisti su
Spotify, con un focus particolare sulla scena musicale italiana come caso di
studio principale. L'Italia presenta un ecosistema musicale ricco e
stratificato, caratterizzato da una forte tradizione melodica e cantautorale
che convive con scene emergenti e dinamiche, come l'hip hop e la trap,
particolarmente attive sul fronte delle collaborazioni. La scelta di
focalizzarsi sul contesto italiano è motivata da diverse ragioni. In primo
luogo, consente di esaminare in profondità le dinamiche collaborative
all'interno di un mercato linguistico e culturale ben definito, dove la
condivisione della lingua italiana può agire sia da fattore di coesione che da
potenziale barriera per la diffusione internazionale. In secondo luogo, la
scena musicale italiana contemporanea sta vivendo una fase di forte
trasformazione e ibridazione, con generi un tempo considerati di nicchia che
raggiungono oggi il grande pubblico, spesso proprio attraverso strategie
collaborative. Infine, l'analisi del caso italiano offre l'opportunità di
testare gli strumenti della SNA in un contesto complesso, dove convivono
artisti di caratura internazionale, figure iconiche della storia della musica e
una vivace scena indipendente.

Attraverso una prospettiva di rete, questo lavoro mira a rispondere a domande
fondamentali sul funzionamento dell'ecosistema musicale italiano:
\begin{itemize}[leftmargin=*, itemsep=5pt]
  \item Quali sono le caratteristiche strutturali fondamentali della rete di
        collaborazioni musicali italiane su Spotify?
  \item Chi sono gli artisti più centrali e influenti, e che tipo di ruolo strutturale
        ricoprono?
  \item Esistono comunità ben definite corrispondenti a generi o correnti artistiche?
  \item Quali sono i pattern di collaborazione prevalenti? Gli artisti tendono a
        collaborare con colleghi simili a sé per grado di connessione, popolarità o
        genere musicale?
  \item Come si posiziona e relaziona la scena italiana con il panorama musicale
        internazionale?
\end{itemize}

Per rispondere a questi interrogativi, lo studio applica un'ampia gamma di
metriche e tecniche di analisi di rete a un dataset ricavato da Spotify, opportunamente
arricchito con informazioni su nazionalità e genere degli artisti.

\section{Datasets}
\label{datasets}

The initial dataset used for this analysis was downloaded from
\href{https://www.kaggle.com/datasets/jfreyberg/spotify-artist-feature-collaboration-network}{Spotify
  Artist Feature \& Collaboration Network} \cite{kaggle-dataset}. After careful
inspection, it was considered a reliable source, as it is derived from publicly
available data provided by the Spotify API and already pre-processed to
represent collaboration relationships among artists.

The dataset is structured as a directed graph and consists of two main files:
\begin{itemize}
  \item \textbf{nodes.csv} -- contains the nodes of the graph, where each node represents an artist. The columns include:
        \begin{itemize}
          \item \texttt{id}: unique identifier of the artist.
          \item \texttt{name}: name of the artist.
          \item \texttt{followers}: number of followers of the artist on Spotify.
          \item \texttt{popularity}: popularity index (0--100), computed by Spotify based on recent streams and overall visibility.
          \item \texttt{genres}: list of genres associated with the artist.
          \item \texttt{chart\_hits}: List showing the number of Spotify chart hits in different countries (according to kworb.net)
        \end{itemize}
  \item \textbf{edges.csv} -- contains the edges of the graph, representing a collaboration between two artists. The columns include:
        \begin{itemize}
          \item \texttt{source}: ID of the collaborating artist.
          \item \texttt{target}: ID of the artist being collaborated with.
        \end{itemize}
\end{itemize}

The initial objective was to enrich the graph with additional artist-level
information, specifically \textbf{nationality} and \textbf{dominant musical
  genre}, in order to enable more in-depth social and cultural analyses of the
network.

\subsection{Artist Nationality Enrichment}
\label{subsec:nationality}

To associate a nationality with each artist, two complementary strategies were
adopted:

\begin{enumerate}
  \item \textbf{Inference based on musical genre.}
        In the first approach, nationality was inferred by analyzing the associated musical genres. For instance, an artist labeled with the genre \texttt{italian hip hop} was classified as \emph{Italian}. This method allowed the automatic assignment of nationality to a substantial subset of artists; however, it was not applicable in all cases, as many genres do not contain explicit geographical references.

  \item \textbf{Completion using an external dataset (MusicBrainz).}
        For artists whose nationality could not be inferred in the first step, data from the \texttt{MusicBrainz Dump (mbdump)} \cite{musicbrainz} were integrated. A direct matching based solely on artist names posed significant challenges due to the presence of homonyms with different nationalities. To mitigate this issue, the integration was performed exclusively on artists that remained unclassified after the first inference step, thereby improving overall precision and preserving data consistency.
\end{enumerate}

This hybrid procedure increased the coverage of nationality information and
enabled a more accurate subsequent analysis, particularly when comparing
artistic communities across different countries.

\subsection{Aggiunta del genere artistico}
Per associare uno o più generi musicali agli artisti del dataset, è stata
progettata una procedura con l’obiettivo di massimizzare il numero di artisti
con genere musicale assegnato, mantenendo al contempo coerenza e plausibilità
semantica delle assegnazioni. Il processo si articola in più fasi successive,
ciascuna delle quali interviene solo sugli artisti rimasti privi di genere nel
passaggio precedente.

\begin{itemize}
  \item \textbf{Mappatura diretta dei generi Spotify.}
        In una prima fase, i generi specifici forniti da Spotify (ad esempio \emph{italian pop}, \emph{alternative rock}, \emph{deep house}) sono stati normalizzati e ricondotti a un insieme limitato di \emph{macro-categorie} musicali (come \emph{Pop}, \emph{Rock}, \emph{Hip Hop / Rap}, \emph{Elettronica / Dance}, ecc.). Questa mappatura è stata realizzata tramite regole lessicali e keyword-based, consentendo di ridurre l’elevata frammentazione dei generi originali e di ottenere una rappresentazione più compatta e comparabile.

  \item \textbf{Completamento assistito tramite AI.}
        I generi che non risultavano mappabili automaticamente nella fase precedente (raccolti nella categoria \emph{Altri / Specifici}) sono stati estratti e forniti come input a un processo di classificazione assistito da intelligenza artificiale. L’AI ha ricondotto ciascun genere residuo a una delle macro-categorie musicali precedentemente definite, sulla base di similarità semantiche e conoscenza musicale generale. Le associazioni così ottenute sono state successivamente reintegrate nel dataset, consentendo di ridurre in modo significativo il numero di generi non classificati e di migliorare la copertura complessiva della mappatura, mantenendo coerenza con lo schema di categorizzazione adottato.

  \item \textbf{Inferenza tramite collaborazioni artistiche.}
        Per gli artisti privi di genere dopo la mappatura diretta, è stato sfruttato il grafo delle collaborazioni. In particolare, sono stati assegnati i generi più frequenti tra i collaboratori diretti; qualora ciò non fosse sufficiente, l’inferenza è stata estesa tramite una ricerca BFS (Breadth-First Search) fino a tre livelli di distanza nella rete, selezionando i generi più ricorrenti nei nodi visitati.

  \item \textbf{Inferenza basata su metriche di popolarità.}
        Gli artisti ancora non classificati sono stati analizzati in base a indicatori quantitativi come numero di follower e popolarità. Attraverso semplici euristiche derivate da pattern osservati nel dataset (ad esempio alta popolarità associata a generi mainstream), sono stati inferiti i generi più probabili.

  \item \textbf{Assegnazione di fallback globale.}
        Infine, per i rari casi rimasti senza genere, è stato applicato un meccanismo di fallback basato sui generi globalmente più comuni nel dataset, garantendo che ogni artista fosse associato ad almeno una macro-categoria musicale.

\end{itemize}

Questa strategia multilivello ha permesso di ottenere un dataset completo e
consistente dal punto di vista dei generi musicali, riducendo al minimo le
assegnazioni arbitrarie e sfruttando in modo integrato informazioni semantiche,
strutturali e quantitative. Il risultato finale costituisce una base solida per
le successive analisi di rete e di comunità artistiche.

\section{Validity and Reliability}
\label{validity-and-reliability-not-needed-for-the-project-proposal}

The initial dataset is derived from the Spotify API and pre-processed to
explicitly model collaboration relationships among artists, providing a
reasonable approximation of actual musical interactions on the platform.
However, the dataset reflects Spotify’s ecosystem and temporal snapshot, and
therefore may not capture collaborations occurring outside the platform or
informal artistic relationships.

Additional artist-level attributes, such as nationality and dominant musical
genre, were introduced to support higher-level social and cultural analyses.
Nationality was inferred using a hybrid approach combining genre-based cues and
external data from MusicBrainz, applied selectively to reduce ambiguity due to
artist name homonyms. While this procedure increases coverage and interpretive
power, it introduces a degree of uncertainty, particularly for artists whose
identity or geographic origin is weakly signaled by available metadata.

Similarly, musical genres were consolidated into a limited set of
macro-categories through a multi-step process involving rule-based mapping,
AI-assisted classification, and network-based inference. This approach improves
comparability and completeness, but necessarily abstracts away finer-grained
genre distinctions and may propagate local biases through collaboration-based
inference.

Regarding reliability, all data processing and enrichment steps follow
deterministic rules or documented heuristics, ensuring that the analysis is
reproducible given the same inputs and parameters. External data sources and
AI-assisted mappings represent potential sources of variability; however, their
use was constrained to well-defined stages and applied consistently across the
dataset. Overall, the adopted methodology yields a dataset that is both valid
and reliable for social network analysis, as it is grounded in authoritative
data sources, enriched through controlled and well-documented procedures, and
constructed to balance descriptive accuracy with reproducibility, making it a
solid foundation for the subsequent network and community analyses.

\section{Misure e Risultati}
\label{measures}
In questa sezione si riassumono in modo sintetico le principali misure utilizzate, le tecnologie impiegate e il loro legame con gli obiettivi dello studio.
\subsection*{Rappresentazione della rete}
\begin{itemize}
  \item Grafo non orientato $G = (V, E)$: nodi = artisti (\texttt{spotify\_id}), archi
        = collaborazioni tra artisti presenti nelle tracce.
  \item Implementazione in Python con \texttt{pandas} per i CSV dei nodi/archi e
        \texttt{NetworkX} per la costruzione del grafo e il calcolo delle misure.
\end{itemize}

\subsection*{Misure di centralità}
\begin{itemize}
  \item \textbf{Degree centrality}: normalizza il numero di collaborazioni di ciascun artista, identifica gli hub più connessi e viene usata per selezionare i top artisti nel sottografo di analisi.
  \item \textbf{Betweenness centrality}: misura quante volte un artista cade sui cammini minimi tra coppie di nodi, individuando i ``broker'' strutturali tra comunità e generi diversi.
  \item \textbf{Closeness centrality}: inverso della distanza media da un artista a tutti gli altri, quantifica quanto rapidamente un artista può raggiungere il resto della rete.
  \item \textbf{Eigenvector centrality}: assegna punteggi più alti agli artisti collegati ad altri artisti centrali, catturando l’appartenenza al ``core'' della scena.
\end{itemize}

\subsection*{Community detection e bridge}
\begin{itemize}
  \item \textbf{Louvain}: individua comunità massimizzando la modularità, permettendo di associare cluster strutturali a macro-generi, scene nazionali o gruppi di etichetta.
  \item \textbf{Edge betweenness} e \textbf{constraint} di Burt: identificano rispettivamente collaborazioni-ponte tra comunità e artisti con accesso a \textit{structural holes}, fondamentali per la diffusione di stili e contenuti tra mondi diversi.
\end{itemize}

\subsection*{Generi, nazionalità e successo}
\begin{itemize}
  \item Generi e nazionalità sono gestiti come attributi dei nodi (\texttt{genre},
        \texttt{nationality}); si contano collaborazioni intra/inter-genere e
        intra/inter-nazionali per valutare assortatività e aperture transnazionali.
  \item Per gli artisti senza genere, il genere viene inferito dal genere più frequente
        nel vicinato di rete, con soglia minima di collaborazioni per garantire
        robustezza.
  \item Le misure strutturali sono correlate con indicatori esterni
        (\texttt{popularity} Spotify, numero collaborazioni, collaborazioni estere,
        presenza in chart) per studiare il legame tra posizione nella rete, popolarità
        ed espansione internazionale.
\end{itemize}

\subsection*{Artisti emergenti}
\begin{itemize}
  \item Si costruisce un \texttt{DataFrame} con \texttt{popularity} e numero di
        collaborazioni per artista; soglie su entrambi gli indicatori definiscono tre
        classi: \emph{emergente}, \emph{intermedio}, \emph{affermato}.
  \item La matrice delle collaborazioni tra classi (emergente–emergente,
        emergente–affermato, ecc.) mostra le strategie di networking (orizzontale tra
        pari vs collegamento verso artisti affermati) e come queste si riflettano nella
        crescita di centralità e popolarità.
\end{itemize}
\subsection{Analisi generale della rete di collaborazioni musicali italiane}

Al fine di delineare le differenze strutturali chiave e ottenere una panoramica
della topologia e delle dinamiche interne della rete di collaborazioni musicali
italiane, è stata calcolata una serie di misure generali, come riportato nella
Tabella \ref{tab:metrics_italia}.

Il \textbf{numero di componenti connesse} è stato misurato per identificare la
presenza di sottogruppi isolati all'interno dell'ecosistema musicale nazionale.
Sia il \textbf{grado massimo} che il \textbf{grado medio} dei nodi sono stati
calcolati per valutare la connettività globale della rete e per quantificare
l'intensità dell'attività collaborativa degli artisti. La \textbf{densità}
della rete è stata calcolata per misurare la sua coesione complessiva. Il
\textbf{diametro} e la \textbf{lunghezza media del percorso più breve}
forniscono una valutazione dell'efficienza del flusso di informazioni e
indicano la facilità con cui gli artisti possono essere raggiunti gli uni
attraverso gli altri. Infine, il \textbf{coefficiente di clustering medio} e la
\textbf{transitività} sono stati calcolati per misurare la coesione locale
della rete e la tendenza alla formazione di gruppi strettamente connessi.

\begin{table}[h]
  \centering
  \caption{Caratteristiche strutturali della rete di collaborazioni musicali italiane}
  \label{tab:metrics_italia}
  \begin{tabular}{|l|c|}
    \hline
    \textbf{Parametro}                     & \textbf{Valore} \\
    \hline
    Nodi totali (artisti)                  & 1.656           \\
    Archi totali (collaborazioni)          & 4.307           \\
    Componenti connesse                    & 16              \\
    Grado massimo del nodo                 & 114             \\
    Grado medio del nodo                   & 5.20            \\
    Densità                                & 0.00314         \\
    Diametro                               & 10              \\
    Lunghezza media del percorso più breve & 4.14            \\
    Coefficiente di clustering medio       & 0.119           \\
    Transitività                           & 0.128           \\
    \hline
  \end{tabular}
\end{table}

I sedici componenti connessi suggeriscono l'esistenza di gruppi isolati,
corrispondenti a nicchie o comunità artistiche con contatti limitati con il
resto dell'ecosistema nazionale. Ciascun artista è connesso, in media, a circa
cinque colleghi nella rete. Tuttavia, la distribuzione del grado è fortemente
eterogenea: la presenza di un nodo con grado 114 rivela un \textit{hub} di
importanza centrale, mentre il 54.6\% degli artisti presenta un grado pari a 1,
collaborando dunque con un solo altro artista. La densità estremamente bassa
(circa lo 0.31\% delle possibili connessioni è realizzata) conferma la natura
\textbf{sparsa} della rete. Il diametro della rete è pari a 10, indicando che
due artisti qualsiasi possono essere collegati attraverso al massimo 10
collaborazioni intermedie. Nonostante la bassa densità, la rete presenta una
lunghezza media del percorso contenuta (4.14), indicando che gli artisti sono
collegati attraverso poche collaborazioni intermedie.

I valori del coefficiente di clustering medio (0.119) e della transitività
(0.128) sono moderati e tra loro vicini. Ciò indica una tendenza misurabile,
seppur non dominante, alla \textbf{chiusura triadica}: due collaboratori di uno
stesso artista hanno una probabilità di circa il 12\% di aver collaborato a
loro volta. Questa coesione locale favorisce la formazione di circoli artistici
coesi e comunità parzialmente clusterizzate, contribuendo alla stabilità delle
relazioni collaborative e alla condivisione di pratiche artistiche all'interno
di sottogruppi, pur mantenendo sufficiente apertura per permettere connessioni
tra comunità diverse.

\subsection{Analisi Comparativa con i Principali Paesi Europei}

Per collocare i risultati italiani in un contesto continentale più ampio,
l'analisi è stata estesa a tutti i principali paesi europei. Questo approccio
consente una valutazione comparativa completa delle dinamiche collaborative.

La Tabella \ref{tab:comparative_results_eu} presenta il confronto tra l'Italia
e i quattro principali paesi europei per dimensione della rete.

\begin{table}[h]
  \centering
  \caption{Confronto con i Principali Paesi Europei}
  \label{tab:comparative_results_eu}
  \begin{tabular}{|l|c|c|c|c|c|}
    \hline
    \textbf{Metrica}     & \textbf{Italia} & \textbf{Francia} & \textbf{Germania} & \textbf{Regno Unito} & \textbf{Paesi Bassi} \\
    \hline
    Total nodes          & 1,656 (3°)      & 1,643 (4°)       & 2,706 (2°)        & 3,290 (1°)           & 1,420 (5°)           \\
    Total edges          & 4,307 (5°)      & 4,754 (4°)       & 5,929 (2°)        & 7,532 (1°)           & 5,143 (3°)           \\
    Average degree       & 5.20 (5°)       & 5.79 (2°)        & 4.38 (7°)         & 4.58 (6°)            & 7.24 (1°)            \\
    Connected components & 16 (4°)         & 27 (6°)          & 32 (7°)           & 70 (9°)              & 12 (3°)              \\
    Average clustering   & 0.119 (4°)      & 0.113 (6°)       & 0.120 (3°)        & 0.062 (9°)           & 0.151 (2°)           \\
    \hline
  \end{tabular}
\end{table}

Dall'analisi emergono differenze strutturali significative tra i paesi,
evidenziando modelli distinti di collaborazione musicale:
\begin{enumerate}
  \item \textbf{Paesi Bassi}: Presentano l'\textbf{average degree} più alto d'Europa (7.24) nonostante siano solo quinti per numero di artisti. Il loro \textbf{average clustering coefficient} (0.151) è il secondo più alto, e la percentuale di hub artists (17.8\%) è la massima in Europa.

  \item \textbf{Polonia}: Con un \textbf{average degree} di 6.20 e \textbf{average clustering coefficient} di 0.167, rappresentano un modello di rete altamente coesa e interconnessa. Solo 8 \textbf{connected components} indicano un'eccellente integrazione strutturale.

  \item \textbf{Grecia}: Ha il \textbf{average clustering coefficient} più alto d'Europa (0.183) e un ottimo \textbf{average degree} (5.47). Con soli 4 \textbf{connected components}, è una delle reti meglio integrate.

  \item \textbf{Regno Unito}: Nonostante le dimensioni maggiori (3.290 artisti), presenta il \textbf{average clustering coefficient} più basso (0.062) e la frammentazione più alta (70 \textbf{connected components}), riflettendo un mercato vasto ma segmentato.
        Presenta anche una densità di rete significativamente più bassa
        (0.001392) rispetto a quella osservata per l'Italia (0.003143). Tale differenza
        riflette due configurazioni strutturali distinte: da un lato, una rete
        britannica estesa ma caratterizzata da un'elevata dispersione delle relazioni
        collaborative; dall'altro, un ecosistema italiano di dimensioni più contenute
        ma relativamente più denso e coeso.

\end{enumerate}

\subsubsection{Analisi Multidimensionale e Interpretazione dei Pattern Europei}

L'analisi multidimensionale sintetizza tre aspetti critici della struttura
delle reti europee: dimensione, intensità collaborativa e relazione tra queste
variabili.

\begin{figure}[htbp]
  \centering
  \includegraphics[width=0.95\textwidth]{network_size_eu.png}
  \caption{Analisi multidimensionale comparativa delle reti di collaborazione musicale a livello europeo.}
  \label{fig:network_analysis_eu}
\end{figure}

Gli istogrammi presentano i primi 15 paesi europei per due metriche distinte: a
sinistra il numero di artisti, al centro il numero totale di collaborazioni,
nel quale si notano:

\begin{itemize}
  \item Il \textbf{Regno Unito} (7.532 collaborazioni) e la \textbf{Germania} (5.927)
        si collocano ai vertici per numero complessivo di collaborazioni; tale
        risultato è principalmente riconducibile alla loro elevata numerosità di nodi.

  \item I \textbf{Paesi Bassi} (5.143 collaborazioni) rappresentano il caso più
        rilevante: pur essendo quinti per numero di artisti, si posizionano terzi per
        volume di collaborazioni, superando sia l'Italia (4.307) che la Francia
        (4.754). Ciò indica un'intensa attività collaborativa.

  \item La \textbf{Polonia} (3.391 collaborazioni) mostra un'intensità sorprendente,
        superando paesi con reti più estese come Svezia (2.644) e Finlandia (2.329).
\end{itemize}

\subsubsection{Relazione Strutturale tra Artisti e Collaborazioni}

Il grafico a destra esplora la relazione fondamentale tra la dimensione di una
rete (numero di artisti) e la sua attività (numero di collaborazioni). La
relazione tra il numero di artisti e il numero totale di collaborazioni è stata
analizzata mediante una regressione lineare stimata con il metodo dei minimi
quadrati, considerando tutti i paesi europei inclusi nello studio. La retta
ottenuta, \( y = 2.44x - 76 \), descrive l'andamento medio delle reti musicali
europee.

\begin{itemize}
  \item \textbf{Coefficiente angolare (2.44)}: indica che, in media, a ogni
        artista aggiuntivo corrispondono circa 2.44 collaborazioni in più.
        Questo valore esprime l'intensità collaborativa media dei network
        musicali analizzati.

  \item \textbf{Intercetta (-76)}: il valore negativo suggerisce che reti molto
        piccole tendono ad avere un numero di collaborazioni inferiore a quello
        previsto dalla tendenza generale. In particolare, al di sotto di circa
        30 artisti, la struttura della rete risulta meno attiva.
\end{itemize}

La retta di regressione può quindi essere utilizzata come riferimento: i paesi
posizionati al di sopra mostrano un livello di collaborazione superiore a
quanto atteso rispetto alle loro dimensioni, mentre quelli al di sotto
presentano un'attività collaborativa più contenuta.

\begin{itemize}
  \item I paesi che si collocano \textbf{sopra la retta di regressione} (Paesi Bassi,
        Polonia, Grecia) rappresentano ecosistemi iper-collaborativi, dove il volume di
        interazioni supera sistematicamente le aspettative date le dimensioni. Questi
        sistemi sono tipicamente caratterizzati da alti valori di densità e
        coefficiente di clustering.

  \item I paesi \textbf{prossimi alla retta} (Italia, Francia, Germania, Regno Unito)
        seguono una relazione approssimativamente lineare, mostrando una scalabilità
        ``normativa'' dell'attività collaborativa in funzione della dimensione.

  \item La \textbf{dispersione} dei dati conferma l'assenza di un modello unico
        europeo, evidenziando invece una pluralità di configurazioni strutturali.
\end{itemize}

\subsection{Analisi Comparativa con i Principali Paesi Non-Europei}

Per collocare l'ecosistema musicale italiano in un contesto globale più ampio,
è stata condotta un'analisi comparativa estesa a 49 paesi non-europei con reti
collaborative significative (su 158 paesi analizzati).

La Tabella \ref{tab:comparative_results_global} presenta il confronto tra
l'Italia e i principali paesi non-europei.

\begin{table}[h]
  \centering
  \caption{Confronto con i Principali Paesi Non-Europei}
  \label{tab:comparative_results_global}
  \begin{tabular}{|l|c|c|c|c|c|}
    \hline
    \textbf{Metrica}     & \textbf{Italia} & \textbf{Stati Uniti} & \textbf{Brasile} & \textbf{India} & \textbf{Messico} \\
    \hline
    Total nodes          & 1,656           & 6,217 (1°)           & 1,859 (2°)       & 999 (4°)       & 773 (5°)         \\
    Total edges          & 4,307           & 14,860 (1°)          & 5,547 (2°)       & 2,834 (4°)     & 1,665 (5°)       \\
    Average degree       & 5.20 (6°)       & 4.78 (7°)            & 5.97 (2°)        & 5.67 (3°)      & 4.31 (8°)        \\
    Connected components & 16              & 111 (1°)             & 3 (9°)           & 2 (10°)        & 4 (8°)           \\
    Average clustering   & 0.119           & 0.082                & 0.173 (3°)       & 0.181 (2°)     & 0.156 (4°)       \\
    \hline
  \end{tabular}
\end{table}
Dall'analisi emergono differenze strutturali significative tra i paesi,
evidenziando modelli distinti di collaborazione musicale:

\begin{enumerate}
  \item \textbf{Stati Uniti}: Rete di dimensioni eccezionali (6.217 artisti, 14.860 collaborazioni) ma con basso \textbf{average clustering coefficient} (0.082) e alta frammentazione (111 \textbf{connected components}).

  \item \textbf{Brasile}: Con 1.859 artisti e solo 3 \textbf{connected components}, rappresenta un modello di rete altamente integrata. L'\textbf{average degree} di 5.97 e il \textbf{average clustering coefficient} di 0.173 indicano un ecosistema vibrante e coeso.

  \item \textbf{India}: Nonostante dimensioni più contenute (999 artisti), presenta caratteristiche strutturali eccellenti: solo 2 \textbf{connected components}, \textbf{average degree} di 5.67, e \textbf{average clustering coefficient} di 0.181.

  \item \textbf{Porto Rico}: Caso estremo con \textbf{average degree} di 10.66 (il più alto al mondo) e \textbf{average clustering coefficient} di 0.314, indicando una rete ultra-collaborativa e ultra-coesa.
\end{enumerate}

L'analisi rivela una marcata asimmetria dimensionale tra i sistemi musicali
analizzati:

\begin{itemize}
  \item Gli \textbf{Stati Uniti} presentano caratteristiche dimensionali eccezionali,
        con una rete di 6.217 artisti e 14.860 collaborazioni, oltre 3,7 volte più
        grande della rete italiana. Questa dimensione considerevole è accompagnata da
        una bassa densità (0.000769) che riflette l'ampia scala e diversificazione del
        mercato musicale statunitense, con ben 111 gruppi distinti che operano in modo
        relativamente isolato. Il grado medio (4.78) è leggermente inferiore a quello
        italiano (5.20), indicando che, nonostante le dimensioni maggiori, la rete
        statunitense mostra una minore intensità di connessioni per artista.

  \item Il \textbf{Brasile} (1.859 artisti) e l'\textbf{India} (999 artisti)
        rappresentano i maggiori sistemi musicali emergenti, entrambi caratterizzati da
        reti ad alta densità (0.003212 e 0.005685 rispettivamente) e da un'elevata
        coesione strutturale. Questi valori indicano ecosistemi musicali altamente
        integrati, con soli 3 e 2 gruppi distinti rispettivamente, che riflettono una
        forte unità culturale e geografica.

  \item L'Italia si posiziona in una fascia intermedia a livello globale, con
        dimensioni paragonabili a quelle di Giappone (890) e Messico (773). Rispetto a
        questi paesi, l'Italia mostra una struttura più coesa: il Giappone, pur avendo
        più artisti, presenta una rete più frammentata (38 gruppi distinti) e un grado
        medio inferiore (2.48), mentre il Messico, sebbene abbia meno artisti, mostra
        una densità superiore (0.005580) e un'elevata integrazione strutturale.
\end{itemize}

L'analisi identifica diversi modelli di ecosistemi musicali caratterizzati da
intensa attività collaborativa:

\begin{itemize}
  \item \textbf{Porto Rico} rappresenta il caso più marcato di iper-collaboratività, con un grado medio particolarmente elevato (10.66) e un coefficiente di clustering alto (0.314). Questo indica non solo che gli artisti portoricani collaborano molto tra loro, ma che i loro collaboratori tendono a collaborare a loro volta, formando una rete altamente integrata e interconnessa. La densità relativamente alta (0.032614) conferma questa compattezza.

  \item Il \textbf{Brasile} e l'\textbf{India} mostrano strutture simili, con gradi
        medi elevati (5.97 e 5.67 rispettivamente) e clustering significativo (0.173 e
        0.181). Questi valori suggeriscono ecosistemi vibranti con forte tendenza alla
        formazione di comunità coese, supportata da una bassa frammentazione (3 e 2
        gruppi distinti) che indica scene musicali unificate e ben integrate.

  \item La \textbf{Corea del Sud} presenta un caso interessante di rete moderatamente
        grande (708 artisti) ma con alta densità (0.006185) e grado medio significativo
        (4.37), riflettendo una scena musicale strutturata e interconnessa,
        probabilmente influenzata dal sistema dell'industria musicale K-pop che
        favorisce collaborazioni frequenti.

  \item L'Italia, con un grado medio di 5.20, si colloca tra i paesi con maggiore
        attività collaborativa a livello globale, superando paesi come Germania (4.38),
        Regno Unito (4.58) e Corea del Sud (4.37). Questo posizionamento indica una
        cultura musicale particolarmente orientata alla collaborazione, che compensa le
        dimensioni più contenute della rete con una maggiore intensità di interazioni.
\end{itemize}

\subsubsection{Densità e frammentazione}

\begin{table}[h]
  \centering
  \caption{Analisi Comparativa della Densità e Frammentazione}
  \label{tab:density_fragmentation}
  \begin{tabular}{|l|c|c|c|}
    \hline
    \textbf{Tipo di Rete}           & \textbf{Densità Tipica} & \textbf{Gruppi Distinti} & \textbf{Esempi}                  \\
    \hline
    Reti Piccole Ultra-dense        & 0.10-1.00               & 1-5                      & Ghana, Egitto, Cuba              \\
    Reti Medie Dense                & 0.01-0.10               & 2-10                     & Colombia, Argentina, Sudafrica   \\
    Reti Grandi Moderatamente dense & 0.001-0.01              & 10-50                    & Italia, Brasile, India           \\
    Reti Molto Grandi Sparse        & <0.001                  & >50                      & Stati Uniti, Giappone, Australia \\
    \hline
  \end{tabular}
\end{table}

\begin{itemize}
  \item I paesi \textbf{africani} rappresentano il modello delle reti piccole
        ultra-dense: Ghana (densità 0.138), Egitto (0.110), Nigeria (0.047). Queste
        reti, pur avendo pochi artisti (24-66 nodi), mostrano un'elevatissima
        interconnessione interna. Questo suggerisce scene musicali fortemente coese e
        geograficamente concentrate, dove la maggior parte degli artisti collabora
        direttamente o indirettamente.

  \item Le reti \textbf{latinoamericane} mostrano densità intermedie (0.01-0.10) con
        buona integrazione: Colombia (0.010), Argentina (0.010), Messico (0.006).
        Questi valori riflettono scene musicali regionalmente coese ma sufficientemente
        ampie da sostenere diversificazione interna, con un numero limitato di gruppi
        distinti (2-5).

  \item L'Italia si colloca tra le reti moderatamente dense (0.00314), tipiche delle
        scene musicali europee mature che bilanciano dimensione e coesione. La densità
        italiana è simile a quella francese (0.00352) e superiore a quella tedesca
        (0.00162) e britannica (0.00139), indicando una maggiore compattezza
        strutturale.
\end{itemize}

\subsubsection{Clustering e coesione Locale}

\begin{itemize}
  \item I più alti valori di clustering si osservano in \textbf{Porto Rico} (0.314),
        \textbf{Ghana} (0.287), e \textbf{Repubblica Dominicana} (0.275), indicando una
        forte tendenza alla formazione di ``triadi chiuse'' dove i collaboratori di un
        artista collaborano frequentemente tra loro. Questi valori particolarmente alti
        suggeriscono comunità musicali estremamente coese, spesso legate a specifici
        generi o scene locali.

  \item L'\textbf{India} (0.181) e il \textbf{Brasile} (0.173) mostrano valori di
        clustering elevati, riflettendo scene musicali che combinano dimensioni
        significative con forte coesione interna, probabilmente legata a identità
        culturali e linguistiche forti.

  \item L'Italia (0.119) presenta un valore di clustering moderato ma significativo,
        superiore a quello di molti paesi europei (Francia 0.113, Germania 0.121,
        Spagna 0.088, Regno Unito 0.062) e paragonabile a quello di alcune scene
        emergenti. Questo indica una tendenza alla formazione di circoli artistici
        coesi pur mantenendo sufficiente apertura per nuove connessioni.

  \item I bassi valori di clustering in paesi come \textbf{Canada} (0.037),
        \textbf{Australia} (0.035), e \textbf{Cina} (0.027) suggeriscono strutture più
        ``a stella'' o meno inclini alla chiusura triadica, possibilmente influenzate
        da fattori geografici, linguistici o industriali che limitano la formazione di
        comunità strettamente interconnesse.
\end{itemize}

\subsection{Nodes}
\subsubsection{Misure di centralità}

Per identificare gli artisti strutturalmente più importanti nella rete di
collaborazioni, sono state calcolate quattro misure di centralità. L'obiettivo
è comprendere quali artisti occupano posizioni strategiche e come queste
posizioni si manifestano attraverso diversi aspetti della struttura della rete.

\paragraph{Degree Centrality}

La \textbf{degree centrality} quantifica il numero di connessioni dirette di un
nodo, normalizzato per il massimo numero possibile di connessioni. Nella rete
musicale analizzata, questa misura rappresenta il numero di artisti diversi con
cui un determinato artista ha collaborato.

L'implementazione calcola sia il grado assoluto, che la degree centrality
normalizzata. I risultati rivelano una distribuzione fortemente asimmetrica: la
degree centrality media è pari a 0.0031, mentre il valore massimo raggiunge
0.0689. Tale distribuzione indica che la maggior parte degli artisti mantiene
un numero limitato di collaborazioni, mentre un ristretto gruppo di nodi
concentra un numero significativamente elevato di connessioni.

La Tabella \ref{tab:degree-centrality} presenta i dieci artisti con la degree
centrality più elevata:

\begin{table}[h]
  \centering
  \caption{Top 10 artisti per Degree Centrality}
  \label{tab:degree-centrality}
  \begin{tabular}{c l r}
    \hline
    \textbf{Posizione} & \textbf{Artista} & \textbf{Degree Centrality} \\
    \hline
    1                  & Guè              & 0.0689                     \\
    2                  & Andrea Bocelli   & 0.0622                     \\
    3                  & Clementino       & 0.0508                     \\
    4                  & Gemitaiz         & 0.0489                     \\
    5                  & Night Skinny     & 0.0483                     \\
    6                  & Don Joe          & 0.0477                     \\
    7                  & Inoki            & 0.0441                     \\
    8                  & Fabri Fibra      & 0.0435                     \\
    9                  & Emis Killa       & 0.0429                     \\
    10                 & Ennio Morricone  & 0.0411                     \\
    \hline
  \end{tabular}
\end{table}

Guè emerge come il nodo più connesso della rete con 114 collaborazioni
distinte, identificandosi come hub centrale della scena musicale italiana. La
presenza di Andrea Bocelli al secondo posto con 103 collaborazioni risulta
particolarmente significativa: nonostante operi in un genere musicale
sostanzialmente diverso (classica/pop crossover), ha sviluppato una rete estesa
di collaborazioni che attraversa molteplici generi musicali. La predominanza di
artisti hip hop nelle prime posizioni (Clementino, Gemitaiz, Night Skinny, Don
Joe, Inoki, Fabri Fibra, Emis Killa) conferma che questo genere presenta
un'elevata propensione alle collaborazioni e costituisce un elemento centrale
nella struttura della rete. La presenza di Ennio Morricone in decima posizione
introduce un elemento di diversità generazionale e stilistica nella classifica.

\paragraph{Eigenvector Centrality}

La \textbf{eigenvector centrality} attribuisce importanza non solo alla
quantità di connessioni, ma alla loro qualità: un artista presenta alta
eigenvector centrality se risulta connesso ad altri artisti che a loro volta
occupano posizioni centrali nella rete. L'algoritmo converge iterativamente,
assegnando a ciascun nodo un punteggio proporzionale alla somma dei punteggi
dei suoi nodi adiacenti.

I risultati mostrano una concentrazione ancora più marcata rispetto alla degree
centrality: il valore medio è 0.0089 mentre il massimo raggiunge 0.2573,
evidenziando che un numero ristretto di artisti forma un nucleo centrale
altamente coeso.

La Tabella \ref{tab:eigenvector-centrality} presenta i dieci artisti con la
eigenvector centrality più elevata:

\begin{table}[h]
  \centering
  \caption{Top 10 artisti per Eigenvector Centrality}
  \label{tab:eigenvector-centrality}
  \begin{tabular}{c l r}
    \hline
    \textbf{Posizione} & \textbf{Artista} & \textbf{Eigenvector Centrality} \\
    \hline
    1                  & Guè              & 0.2573                          \\
    2                  & Gemitaiz         & 0.2069                          \\
    3                  & Emis Killa       & 0.1904                          \\
    4                  & Night Skinny     & 0.1903                          \\
    5                  & Fabri Fibra      & 0.1717                          \\
    6                  & Marracash        & 0.1677                          \\
    7                  & Jake La Furia    & 0.1664                          \\
    8                  & Don Joe          & 0.1650                          \\
    9                  & MadMan           & 0.1533                          \\
    10                 & Lazza            & 0.1425                          \\
    \hline
  \end{tabular}
\end{table}

Guè mantiene la posizione dominante con un valore considerevolmente superiore
agli altri artisti (circa 25\% in più rispetto al secondo classificato),
indicando che le sue collaborazioni coinvolgono prevalentemente gli artisti più
centrali della scena. La presenza esclusiva di artisti hip hop in questa
classifica rivela l'esistenza di un nucleo centrale dominato da questo genere.
Artisti quali Marracash, Jake La Furia, Don Joe, MadMan e Lazza formano un core
altamente interconnesso che definisce il centro della rete hip hop italiana. È
significativo notare che Andrea Bocelli, pur presentando un elevato numero di
collaborazioni (secondo in degree centrality), risulta assente da questa
classifica, suggerendo che le sue collaborazioni coinvolgono prevalentemente
artisti meno centrali o più periferici rispetto al nucleo principale della
rete.

\begin{figure}
  \centering
  \includegraphics[width=0.5\linewidth]{images/eigenvector.pdf}
  \caption{Caption}
  \label{fig:placeholder}
\end{figure}

\paragraph{Closeness Centrality} (NON MI CONVINCE)
La \textbf{closeness centrality} misura la vicinanza di un nodo rispetto a tutti gli altri nodi della rete, calcolando l'inverso della distanza media basata sui cammini minimi. Un artista con elevata closeness centrality può raggiungere rapidamente qualsiasi altro artista nella rete attraverso un numero limitato di intermediari.

I risultati mostrano un valore medio di 0.2336 con un massimo di 0.3677,
indicando che anche gli artisti più centrali necessitano mediamente di circa
tre passaggi per raggiungere qualsiasi altro nodo della rete.

Nel contesto pratico della scena musicale, un'alta closeness centrality
conferisce vantaggi strategici concreti: gli artisti ottimalmente posizionati
possono accedere più facilmente a informazioni e tendenze emergenti in diverse
parti della scena, avviare collaborazioni con artisti distanti nella rete
attraverso pochi intermediari, e diffondere rapidamente il proprio stile o
innovazioni musicali attraverso l'intero ecosistema. Questa metrica identifica
quindi artisti che, indipendentemente dalla loro appartenenza al nucleo
centrale, occupano posizioni che facilitano la comunicazione e la circolazione
di opportunità attraverso l'intera rete.

La Tabella \ref{tab:closeness-centrality} presenta i dieci artisti con la
closeness centrality più elevata:

\begin{table}[h]
  \centering
  \caption{Top 10 artisti per Closeness Centrality}
  \label{tab:closeness-centrality}
  \begin{tabular}{c l r}
    \hline
    \textbf{Posizione} & \textbf{Artista} & \textbf{Closeness Centrality} \\
    \hline
    1                  & Guè              & 0.3677                        \\
    2                  & Clementino       & 0.3576                        \\
    3                  & Gemitaiz         & 0.3537                        \\
    4                  & Fabri Fibra      & 0.3500                        \\
    5                  & J-AX             & 0.3487                        \\
    6                  & Night Skinny     & 0.3480                        \\
    7                  & Marracash        & 0.3466                        \\
    8                  & Emis Killa       & 0.3464                        \\
    9                  & Elisa            & 0.3447                        \\
    10                 & Rocco Hunt       & 0.3420                        \\
    \hline
  \end{tabular}
\end{table}

Guè conferma la sua posizione dominante, risultando ottimalmente posizionato
per accedere all'intera rete con il minimo numero di intermediari. Clementino e
J-AX emergono particolarmente in questa metrica rispetto alla eigenvector
centrality, suggerendo che occupano posizioni strategiche di bridging: pur non
essendo nel core più denso, le loro collaborazioni attraversano diversi cluster
della rete, permettendo loro di fungere da connettori efficienti tra
sottocomunità diverse. La presenza di Elisa (nona posizione) e Rocco Hunt
(decima posizione) introduce una diversificazione di generi significativa.
Questi artisti, pur non appartenendo al nucleo hip hop dominante, mantengono
posizioni di vicinanza globale che facilitano la trasmissione di influenze
cross-genre. La rete, sebbene dominata dall'hip hop nel nucleo centrale,
mantiene quindi una struttura che permette comunicazione efficiente tra generi
diversi, suggerendo un ecosistema musicale dove le barriere tra comunità di
genere non impediscono la circolazione rapida di idee e opportunità
collaborative.

\paragraph{Betweenness Centrality}

La \textbf{betweenness centrality} identifica i nodi che si trovano
frequentemente sui cammini minimi tra altre coppie di nodi. Artisti con elevata
betweenness fungono da intermediari tra diverse componenti della rete, anche
senza necessariamente presentare un numero elevato di connessioni dirette. Il
calcolo è stato approssimato campionando 1000 nodi casuali per ragioni di
efficienza computazionale.

I risultati mostrano una distribuzione fortemente asimmetrica: il valore medio
è 0.0018 mentre il massimo raggiunge 0.1250, indicando che un numero molto
limitato di nodi controlla i flussi di connessione tra cluster diversi della
rete.

Andrea Bocelli domina questa metrica con un valore significativamente superiore
agli altri artisti (quasi 40\% superiore rispetto al secondo classificato),
rivelando il suo ruolo strutturale come ponte tra contesti musicali differenti.
Nonostante non appartenga al core hip hop identificato dalla eigenvector
centrality, Bocelli connette la musica classica e pop con altri generi,
fungendo da intermediario essenziale nella struttura complessiva della rete. La
presenza di Ennio Morricone (quinta posizione), DJ Matrix (sesta posizione),
Jovanotti (nona posizione) e Cristina D'Avena (decima posizione) - artisti che
operano in generi diversi dall'hip hop - conferma che la betweenness centrality
cattura un ruolo strutturale distinto dalla centralità locale. Questi artisti
non appartengono al nucleo centrale ma occupano posizioni di gatekeeping tra
comunità di genere diverse. Clementino mantiene una posizione elevata in tutte
le metriche analizzate (terzo in degree, secondo in closeness e betweenness),
confermandosi come uno degli artisti strutturalmente più rilevanti della rete:
parte integrante del core hip hop ma con spiccate capacità di bridging verso
altre scene musicali.

La Tabella \ref{tab:betweenness-centrality} presenta i dieci artisti con la
betweenness centrality più elevata:

\begin{table}[h]
  \centering
  \caption{Top 10 artisti per Betweenness Centrality}
  \label{tab:betweenness-centrality}
  \begin{tabular}{c l r}
    \hline
    \textbf{Posizione} & \textbf{Artista} & \textbf{Betweenness Centrality} \\
    \hline
    1                  & Andrea Bocelli   & 0.1250                          \\
    2                  & Clementino       & 0.0910                          \\
    3                  & Guè              & 0.0642                          \\
    4                  & Elisa            & 0.0611                          \\
    5                  & Ennio Morricone  & 0.0575                          \\
    6                  & DJ Matrix        & 0.0558                          \\
    7                  & J-AX             & 0.0538                          \\
    8                  & Inoki            & 0.0509                          \\
    9                  & Jovanotti        & 0.0452                          \\
    10                 & Cristina D'Avena & 0.0422                          \\
    \hline
  \end{tabular}
\end{table}

\subsubsection{Assortatività}

L'analisi di assortatività esamina le tendenze di connessione basate su
attributi specifici dei nodi. Sono state calcolate tre forme di assortatività
per caratterizzare i pattern di collaborazione nella rete musicale italiana.

\paragraph{Degree Assortativity}

Il coefficiente di degree assortativity ottenuto è \textbf{-0.1052}, indicando
una rete leggermente \textbf{disassortativa}. Questo valore negativo significa
che artisti con numerose collaborazioni tendono a connettersi con artisti che
ne hanno meno, piuttosto che collaborare prevalentemente tra loro.

Nel contesto della rete musicale analizzata, questo pattern suggerisce che gli
hub (artisti con elevato grado) non formano un gruppo isolato ma includono
nelle loro collaborazioni anche artisti meno prolifici. Questo può riflettere
diverse dinamiche: artisti affermati che forniscono visibilità ad artisti
emergenti, producer e featuring artist che collaborano con un ampio spettro di
artisti a diversi livelli di attività, o più in generale una scena musicale
relativamente aperta dove il numero di collaborazioni pregresse non costituisce
una barriera significativa per future opportunità. Il valore negativo, seppur
moderato, è coerente con una struttura che facilita la mobilità e l'accesso
anche per artisti meno centrali, in contrasto con modelli rigidamente
gerarchici dove esclusivamente le figure di maggior successo collaborano tra
loro.

\paragraph{Followers Assortativity}

Il coefficiente di followers assortativity è \textbf{0.0724}, un valore
positivo ma prossimo allo zero, indicando una tendenza molto debole verso
pattern assortativi basati sulla popolarità.

Questo risultato suggerisce che la popolarità, misurata attraverso il numero di
follower su Spotify, esercita un'influenza limitata sui pattern di
collaborazione. Artisti con un elevato numero di follower non mostrano una
marcata preferenza per collaborazioni esclusive con altri artisti di analoga
popolarità, né emerge una segregazione netta tra diversi strati di popolarità.
L'analisi dettagliata categorizza gli archi in High-High, Low-Low e High-Low
utilizzando la mediana come soglia, rivelando che esistono sia collaborazioni
tra artisti di analoga popolarità che collaborazioni con artisti con un grado
di popolarità diverso dal proprio, in proporzioni relativamente bilanciate.

Questo pattern indica un ecosistema musicale italiano relativamente fluido
rispetto al criterio della fama: le collaborazioni non risultano fortemente
vincolate dal numero di follower, e artisti affermati mostrano disponibilità a
collaborare con colleghi meno conosciuti. Tuttavia, il valore leggermente
positivo suggerisce l'esistenza di una debole preferenza omofila: artisti molto
popolari collaborano con maggiore frequenza con altri artisti popolari, ma
questa tendenza non crea barriere strutturali significative.

\paragraph{Genre Assortativity e Modularity}

I risultati relativi al genere musicale mostrano pattern marcatamente diversi
rispetto alle altre forme di assortatività. Il coefficiente di genre
assortativity è \textbf{0.4778}, un valore decisamente positivo che indica una
forte tendenza assortativa. La modularità è \textbf{Q = 0.2987}, classificabile
come moderata secondo la scala interpretativa implementata (compresa tra 0.1 e
0.3).

Questi valori rivelano che il genere musicale esercita un'influenza
significativa sui pattern di collaborazione: gli artisti tendono a collaborare
prevalentemente con altri artisti dello stesso genere o di generi affini.
Tuttavia, la modularità moderata (non elevata) indica che questa tendenza non
produce compartimenti impermeabili: esistono numerose collaborazioni
cross-genre che mantengono la rete globalmente interconnessa.

La Tabella \ref{tab:genre-pairs} presenta le dieci coppie di generi più
frequenti nelle collaborazioni:

\begin{table}[h]
  \centering
  \caption{Top 10 coppie di generi nelle collaborazioni}
  \label{tab:genre-pairs}
  \begin{tabular}{c l r l}
    \hline
    \textbf{Pos.} & \textbf{Genere 1}           & \textbf{N. collab.} & \textbf{Genere 2} \\
    \hline
    1             & Hip Hop / Rap               & 1829                & (intra-genere)    \\
    2             & Pop                         & 819                 & (intra-genere)    \\
    3             & Hip Hop / Rap               & 596                 & Pop               \\
    4             & Hip Hop / Rap               & 104                 & Indie             \\
    5             & Elettronica / Dance         & 85                  & (intra-genere)    \\
    6             & Classica / Orchestrale      & 72                  & (intra-genere)    \\
    7             & Elettronica / Dance         & 62                  & Pop               \\
    8             & Indie                       & 45                  & Pop               \\
    9             & Soundtrack / Colonne sonore & 37                  & (intra-genere)    \\
    10            & Elettronica / Dance         & 34                  & Hip Hop / Rap     \\
    \hline
  \end{tabular}
\end{table}

L'Hip Hop / Rap domina con 1829 collaborazioni intra-genere, confermando
l'elevata densità e coesione di questa comunità musicale. Il Pop presenta 819
collaborazioni interne, dimostrando anch'esso una significativa coesione
interna. La combinazione Hip Hop / Rap + Pop conta 596 archi, costituendo la
contaminazione cross-genre di gran lunga più frequente. Questo rivela un ponte
strutturale fondamentale tra i due generi dominanti della scena italiana. Le
combinazioni Hip Hop / Rap + Indie (104 archi) e Elettronica / Dance + Pop (62
archi) rappresentano altre contaminazioni significative, sebbene con frequenza
considerevolmente inferiore. Generi quali Classica/Orchestrale (72 archi
interni), Elettronica/Dance (85 archi), e Soundtrack (37 archi) evidenziano
comunità di dimensioni più contenute ma comunque internamente coese.

Il rapporto tra archi intra-genere e inter-genere, calcolato dalla funzione
attraverso l'analisi delle sovrapposizioni tra liste di generi, mostra che le
collaborazioni rimangono prevalentemente all'interno dei confini di genere, con
eccezioni rilevanti.

Nel contesto della scena musicale italiana, questi risultati descrivono un
quadro di specializzazione con contaminazione selettiva: i generi mantengono
identità distinte e gli artisti collaborano prevalentemente entro i propri
confini stilistici, ma esistono ponti consolidati tra generi complementari, in
particolare tra hip hop e pop. L'elevata densità dell'hip hop (1829
collaborazioni interne) conferma quanto emerso dall'analisi di centralità:
questo genere non solo domina il nucleo centrale della rete, ma ha sviluppato
un ecosistema interno estremamente ricco e interconnesso.

\subsection{Network}
\subsubsection{Community Detection}
Community detection analysis was conducted to investigate whether artists tend
to collaborate primarily with other artists belonging to the same musical
macro-genre. To this end, two distinct approaches were applied, namely the
Louvain algorithm and the Edge Betweenness (Girvan--Newman) method. For each
approach, both the number of detected communities and their genre homogeneity
were evaluated by measuring the purity of the dominant macro-genre within each
community.

\noindent
\subsubsection*{Louvain}

\noindent
\begin{minipage}[t]{0.55\textwidth}
  \setlength{\parindent}{0pt}
  The Louvain algorithm identified a total of 34 communities, revealing a
  relatively fragmented network structure. Several communities exhibit a high
  degree of genre homogeneity, particularly for the \emph{Hip Hop / Rap} and
  \emph{Pop} macro-genres, with purity values exceeding 0.6 and reaching 1.0 in
  smaller clusters. At the same time, many communities show a mixed composition,
  with multiple dominant macro-genres coexisting. This behavior is visually
  reflected in the community layout produced by the Louvain algorithm in Gephi
  (Figure~\ref{fig:louvain}), where dense, genre-centered clusters coexist with
  more diffuse, heterogeneous structures. Overall, this indicates that artist
  collaborations are not strictly constrained by genre boundaries, especially
  within larger communities where cross-genre interactions are more frequent.
\end{minipage}
\hfill
\begin{minipage}[t]{0.40\textwidth}
  \centering
  \vspace{0pt}
  \includegraphics[width=\linewidth]{images/louvain.pdf}
  \captionof{figure}{Community structure obtained by applying the Louvain
    algorithm in Gephi.}
  \label{fig:louvain}
\end{minipage}

\subsubsection*{Louvain versus Genre Assortativity}
The community detection results can be interpreted using genre assortativity
and modularity measures. The genre assortativity coefficient
($r = 0.4778$) indicates a clear homophilic tendency, with artists more likely
to collaborate within the same or closely related macro-genres. This behavior
is reflected in the Louvain partition, which identifies high-purity communities,
particularly for \emph{Hip Hop / Rap} and \emph{Pop}.

The moderate modularity value ($Q = 0.2987$) suggests that genre does not
induce a strong structural separation of the network. Several communities
exhibit a mixed genre composition, especially in the Edge Betweenness
partition, where genre purity is often below $0.5$. This pattern is explained
by the presence of numerous cross-genre collaborations, notably between
\emph{Hip Hop / Rap} and \emph{Pop}, which act as structural bridges and limit
overall modularity.

Overall, the network displays a pattern of \emph{selective mixing}: local
genre-based homophily coexists with cross-genre ties, resulting in a structure
that is both cohesive and interconnected.

\subsubsection*{Edge Betweenness}
The Edge Betweenness (Girvan--Newman) algorithm produced 17 communities, resulting in a coarser partitioning of the network compared to the Louvain method. The identified communities are generally less pure, with genre purity values frequently below 0.5, particularly in larger clusters dominated by \emph{Pop} and \emph{Hip Hop / Rap}. This outcome indicates that the iterative removal of highly central edges tends to group together artists from different macro-genres, emphasizing the presence of bridge nodes and inter-genre collaborations rather than a clear separation based on musical genre.

\subsubsection{Degree Distribution}

\noindent
\begin{minipage}[t]{0.55\textwidth}
  \setlength{\parindent}{0pt}
  The analysis of the degree distribution provides insight into the global
  structure of the artist collaboration network. The minimum degree of 1 reflects
  the presence of artists involved in a single collaboration, while the maximum
  degree of 114 highlights a small set of highly connected nodes acting as hubs.
  The average degree of 5.20 indicates an overall sparse network.

  This heterogeneous connectivity pattern is clearly visible in the Gephi
  visualization (Figure~\ref{fig:degree_gephi}), where node size is proportional
  to degree. A large number of small nodes coexist with a few prominent hubs,
  suggesting a strongly right-skewed distribution. This structure is typical of
  complex networks, in which highly connected nodes play a central role in
  maintaining global connectivity and facilitating interactions across different
  regions of the network, potentially spanning multiple musical genres.
\end{minipage}
\hfill
\begin{minipage}[t]{0.40\textwidth}
  \centering
  \vspace{0pt}
  \includegraphics[width=\linewidth]{images/degree.pdf}
  \captionof{figure}{Artist collaboration network visualized in Gephi with node
    size proportional to degree.}
  \label{fig:degree_gephi}
\end{minipage}

\vspace{1em}

\begin{figure}[H]
  \centering
  \subfloat[Degree distribution in linear scale.\label{fig:dd_linear}]{
    \includegraphics[width=0.48\textwidth]{images/DD-AC.png}
  }
  \hfill
  \subfloat[Degree distribution in logarithmic scale.\label{fig:dd_log}]{
    \includegraphics[width=0.48\textwidth]{images/DDLG.png}
  }
  \caption{Degree distribution of the artist collaboration network.
    The linear-scale histogram highlights the high concentration of low-degree
    nodes, while the logarithmic-scale representation emphasizes the long-tailed
    behavior induced by a small number of highly connected artists.}
  \label{fig:degree_distribution}
\end{figure}

\section{Conclusion}
\label{conclusion}

Qualitative analysis of the quantitative findings of the study.

\section{Critique}
\label{critique}

Do you think your work solves the problem presented above? To which extent
(completely, what parts)? Why? What could you have done differently to answer
your research problems (e.g., gather data with additional information, build
your model differently, apply alternative measures)?

\begin{thebibliography}{9}

  \bibitem{project-repo}
  \textit{Network Analysis Project Repository},
  GitHub,
  \url{https://github.com/edefbo1/Network_Analysis.git}

  \bibitem{gephi}
  Gephi Consortium,
  \textit{Gephi: an open source graph visualization and analysis software},
  Gephi.org,
  \url{https://gephi.org/}

  \bibitem{networkx}
  \textit{NetworkX Documentation and Tutorial},
  NetworkX.org,
  \url{https://networkx.org/documentation/stable/tutorial.html}

  \bibitem{kaggle-dataset}
  J. Freyberg,
  \textit{Spotify Artist Feature Collaboration Network},
  Kaggle Dataset,
  \url{https://www.kaggle.com/datasets/jfreyberg/spotify-artist-feature-collaboration-network/}

  \bibitem{musicbrainz}
  MusicBrainz Foundation,
  \textit{MusicBrainz Database — Download and Documentation},
  MusicBrainz.org,
  \url{https://musicbrainz.org/doc/MusicBrainz_Database/Download}

\end{thebibliography}

\end{document}
